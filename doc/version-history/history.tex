%%%%%%%%%%%%%%%%%%%%%%%%%%%%%%%%%%%%%%%%%%%%%%%%%%%%%%%%%%%%%%%%%%%%%%
\section{\label{sec:History-Intro}Introduction to HTCondor Versions}
%%%%%%%%%%%%%%%%%%%%%%%%%%%%%%%%%%%%%%%%%%%%%%%%%%%%%%%%%%%%%%%%%%%%%%

This chapter provides descriptions of what features have been added or
bugs fixed for each version of HTCondor.
The first section describes the HTCondor version numbering scheme, what
the numbers mean, and what the different \Term{release series} are.
The rest of the sections each describe a specific release series, and
all the HTCondor versions found in that series.

%%%%%%%%%%%%%%%%%%%%%%%%%%%%%%%%%%%%%%%%%%%%%%%%%%%%%%%%%%%%%%%%%%%%%%
\subsection{\label{sec:Version-Number-Scheme}
HTCondor Version Number Scheme}
%%%%%%%%%%%%%%%%%%%%%%%%%%%%%%%%%%%%%%%%%%%%%%%%%%%%%%%%%%%%%%%%%%%%%%

Starting with version 6.0.1, HTCondor adopted a new, hopefully easy to
understand version numbering scheme.
It reflects the fact that HTCondor is both a production system and a
research project.
The numbering scheme was primarily taken from the Linux kernel's
version numbering, so if you are familiar with that, it should seem
quite natural.

There will usually be two HTCondor versions available at any given time,
the \Term{stable} version, and the \Term{development} version.
Gone are the days of ``patch level 3'', ``beta2'', or any other random
words in the version string.
All versions of HTCondor now have exactly three numbers, separated by
``.''   

\begin{itemize}

\item The first number represents the major version number, and will
change very infrequently.

\item \emph{The thing that determines whether a version of HTCondor is
\Term{stable} or \Term{development} is the second digit.
Even numbers represent stable versions, while odd numbers represent
development versions.}

\item The final digit represents the minor version number, which
defines a particular version in a given release series.

\end{itemize}


%%%%%%%%%%%%%%%%%%%%%%%%%%%%%%%%%%%%%%%%%%%%%%%%%%%%%%%%%%%%%%%%%%%%%%
\subsection{\label{sec:Stable-Series}The Stable Release Series}
%%%%%%%%%%%%%%%%%%%%%%%%%%%%%%%%%%%%%%%%%%%%%%%%%%%%%%%%%%%%%%%%%%%%%%

People expecting the stable, production HTCondor system should download
the stable version, denoted with an even number in the second digit of
the version string.
Most people are encouraged to use this version.  
We will only offer our paid support for versions of HTCondor from the
stable release series.

\emph{On the stable series, new minor version releases will only
be made for bug fixes and to support new platforms.}
No new features will be added to the stable series.
People are encouraged to install new stable versions of HTCondor when
they appear, since they probably fix bugs you care about.
Hopefully, there will not be many minor version releases for any given
stable series.


%%%%%%%%%%%%%%%%%%%%%%%%%%%%%%%%%%%%%%%%%%%%%%%%%%%%%%%%%%%%%%%%%%%%%%
\subsection{\label{sec:Developement-Series}
The Development Release Series}
%%%%%%%%%%%%%%%%%%%%%%%%%%%%%%%%%%%%%%%%%%%%%%%%%%%%%%%%%%%%%%%%%%%%%%

Only people who are interested in the latest research, new features
that haven't been fully tested, etc, should download the development
version, denoted with an odd number in the second digit of the version
string.  
We will make a best effort to ensure that the development series will
work, but we make no guarantees.

On the development series, new minor version releases will probably
happen frequently.
People should not feel compelled to install new minor versions unless
they know they want features or bug fixes from the newer development
version.

\emph{Most sites will probably never want to install a development
version of HTCondor for any reason.}
Only if you know what you are doing (and like pain), or were
explicitly instructed to do so by someone on the HTCondor Team, should
you install a development version at your site.

After the feature set of the development series is satisfactory to the
HTCondor Team, we will put a code freeze in place, and from that point
forward, only bug fixes will be made to that development series.
When we have fully tested this version, we will release a new stable
series, resetting the minor version number, and start work on a new
development release from there.

%%%%%%%%%%%%%%%%%%%%%%%%%%%%%%%%%%%%%%%%%%%%%%%%%%%%%%%%%%%%%%%%%%%%%%
% The rest of this file just inputs other files which contain sections
% describing each release series in detail.
%%%%%%%%%%%%%%%%%%%%%%%%%%%%%%%%%%%%%%%%%%%%%%%%%%%%%%%%%%%%%%%%%%%%%%

% upgrade instructions are in the Pool Management section
%%%%%%%%%%%%%%%%%%%%%%%%%%%%%%%%%%%%%%%%%%%%%%%%%%%%%%%%%%%%%%%%%%%%%%%
\section{\label{sec:to-8.2}Upgrading from the 8.0 series to the 8.2 series of HTCondor}
%%%%%%%%%%%%%%%%%%%%%%%%%%%%%%%%%%%%%%%%%%%%%%%%%%%%%%%%%%%%%%%%%%%%%%

\index{upgrading!items to be aware of}
Upgrading from the 8.0 series of HTCondor to the 8.2 series 
will bring new features introduced in the 8.1 series of HTCondor.
These new features include:
configuration is more powerful with new syntax and features, and
the default configuration policy does not preempt jobs,
monitoring is enhanced and now integrates with Ganglia,
automated detection and management of GPUs,
numerous scalability enhancements improve performance,
an improved Python API including support for Python 3,
new native packaging and ports are available for the latest Linux 
distributions including Red Hat 7 Beta and Debian 7,
cloud computing improvements including support for
EC2 spot instances, OpenStack, and \Condor{ssh\_to\_job} 
directly into EC2 jobs,
grid universe jobs can now target Google Compute Engine and BOINC servers,
partitionable slots now are compatible with \Condor{startd} \MacroNI{RANK}
expressions, and consumption policies permit partitionable slots 
to be split into dynamic slots at negotiation time,
improved data management including dynamic adjustment of the level of 
file transfer concurrency based on disk load 
(see section~\ref{param:FileTransferDiskLoadThrottle}), 
and experimental support to allow the execution of a job to be overlaid 
with the transfer of output files from the previous different job,
and
the new \Condor{sos} tool helps administrators manage overloaded daemons. 

Upgrading from the 8.0 series of HTCondor to the 8.2 series will
also introduce changes that administrators of sites running from an older
HTCondor version should be aware of when planning an upgrade.  
Here is a list of items that administrators should be aware of.

\begin{itemize}

\item New configuration syntax and features change:
  \begin{itemize}
  \item The interaction of comments and the line continuation character
   has changed.  See  section~\ref{sec:Other-Syntax} for the current
   interaction. 
  \item The use of a colon character (\verb@:@) instead of the
   equals sign (\verb@=@) in assigning a value to a configuration variable
   causes tools that parse configuration to output a warning.
   Therefore, any custom parsing of tool output may need to be updated to
   handle this warning.
   Previous versions of the default configuration set variable
   \MacroNI{RUNBENCHMARKS} using a colon character;
   HTCondor code explicitly suppresses the warning in this case.
  \end{itemize}

\item The default user priority factor for new users has changed 
from 1 to 1000.
Therefore, unless the accountant log is discarded,
existing users will still have a priority factor of 1,
while new users will have a priority factor of 1000.
Use \Condor{userprio} to change the priority factor of existing users
if the accountant log is maintained across the upgrade. 
\Ticket{4282}

\item For Windows platforms,
HTCondor has switched to use the newer 2012 Microsoft compiler,
which uses the Visual C++ 2012 Runtime components.
Therefore, the HTCondor MSI installer will acquire this Runtime,
if it is not already installed.

\item The meaning of \Expr{cpus=auto} when there are more 
slots than CPUs has changed within the configuration. 
In the \Expr{SLOT\_TYPE\_<N>} configuration variable,
\Expr{cpus=auto} previously resulted in 1 CPU per slot. 
Now, all slots with \Expr{cpu=auto} get an equal share of the CPUs, 
rounded down.
\Ticket{3249}

\item The DAGMan node status file formatting has changed.
The format of the DAG node status file is now New ClassAds,
and the amount of information in the file has increased.

\item Setting configuration variable
\Macro{DAGMAN\_ALWAYS\_USE\_NODE\_LOG} to \Expr{False}
or using the corresponding \Opt{-dont\_use\_default\_node\_log} option
to \Condor{submit\_dag} is no longer recommended.
Note that at strictness setting 1 (the default), setting
\MacroNI{DAGMAN\_ALWAYS\_USE\_NODE\_LOG} to \Expr{False}
will cause a fatal error. 
If the DAG must be run with \MacroNI{DAGMAN\_ALWAYS\_USE\_NODE\_LOG} 
set to \Expr{False},
a good way to deal with upgrading is to use DAGMan Halt files 
to cause all of the running DAGs to drain from the queue, 
and then do the upgrade after the DAGs have stopped.  
After the upgrade is done, 
edit the per-DAG configuration files to have 
\MacroNI{DAGMAN\_ALWAYS\_USE\_NODE\_LOG} set to \Expr{True},
or set \MacroNI{DAGMAN\_USE\_STRICT} to 0 and 
re-submit the DAGs, which will then run the Rescue DAGs.

\item If using \Expr{ENABLE\_IPV6 = True}, the configuration must
also set \Expr{ENABLE\_IPV4 = False}.
If both are enabled simultaneously,
daemons will listen on both IPv4 and IPv6, 
but will only advertise one of the two addresses.

\item Globus 5.2.2 or a more recent version is now required 
for grid universe jobs of grid-type nordugrid and cream.
Globus version 5.2.5 is included in this 8.2.0 release of HTCondor.
HTCondor will prefer to use libraries already installed in \File{/usr/lib[64]},
when present.
\Ticket{4243}

\item If referencing attribute \AdAttr{SubmittorUserPrio} in
configuration, such as in the \MacroNI{PREEMPTION\_REQUIREMENTS} expression,
you will need to change it to \AdAttr{SubmitterUserPrio} 
Note the spelling difference in the ClassAd attribute name.
\Ticket{4369}

\item HTCondor can not distinguish normal from abnormal job exit
for Nordugrid ARC grids.
Therefore, all grid-type nordugrid jobs will be recorded as 
terminating normally, with an exit code from 0 to 255.
\Ticket{4342}

\item For configuration, parameter substitution now honors per-daemon 
overrides.  This long standing bug's fix may result in subtle changes
to the way that your configuration files are processed.

\end{itemize}


\input{version-history/upgradingto8-0.tex}
%%%      PLEASE RUN A SPELL CHECKER BEFORE COMMITTING YOUR CHANGES!
%%%      PLEASE RUN A SPELL CHECKER BEFORE COMMITTING YOUR CHANGES!
%%%      PLEASE RUN A SPELL CHECKER BEFORE COMMITTING YOUR CHANGES!
%%%      PLEASE RUN A SPELL CHECKER BEFORE COMMITTING YOUR CHANGES!
%%%      PLEASE RUN A SPELL CHECKER BEFORE COMMITTING YOUR CHANGES!

%%%%%%%%%%%%%%%%%%%%%%%%%%%%%%%%%%%%%%%%%%%%%%%%%%%%%%%%%%%%%%%%%%%%%%
\section{\label{sec:History-8-1}Development Release Series 8.1}
%%%%%%%%%%%%%%%%%%%%%%%%%%%%%%%%%%%%%%%%%%%%%%%%%%%%%%%%%%%%%%%%%%%%%%

This is the development release series of HTCondor.
The details of each version are described below.


%%%%%%%%%%%%%%%%%%%%%%%%%%%%%%%%%%%%%%%%%%%%%%%%%%%%%%%%%%%%%%%%%%%%%%
\subsection*{\label{sec:New-8-1-6}Version 8.1.6}
%%%%%%%%%%%%%%%%%%%%%%%%%%%%%%%%%%%%%%%%%%%%%%%%%%%%%%%%%%%%%%%%%%%%%%

\noindent Release Notes:

\begin{itemize}

\item HTCondor version 8.1.6 released on May 22, 2014.

\end{itemize}


\noindent New Features:

\begin{itemize}

\item HTCondor can discover, schedule, and manage GPUs in an
exceedingly simple way by inserting
\begin{verbatim}
  use feature : GPUs
\end{verbatim}
in the configuration file.
The HTCondor wiki page, 
\URL{https://htcondor-wiki.cs.wisc.edu/index.cgi/wiki?p=HowToManageGpus},
describes the capabilities.

\item The grid universe can now be used to submit and manage jobs on
a BOINC server, using the new grid type \SubmitCmdNI{boinc}.
\Ticket{3540}

\item Configuration has been enhanced in structure and with
newly implemented semantics describing configuration.
As part of this effort, most all configuration variables have
compile-time defaults specified and incorporated into the code.
Therefore, they no longer appear in the example, distributed
configuration file.
It is only when values change that these variables will be placed
into a configuration file.
For current installations wishing to transition to the new, stripped down
configurations files, 
the new \Opt{-writeconfig} option to \Condor{config\_val} will
help to identify values different from defaults.
New configuration semantics permit
\begin{itemize}
  \item the inclusion of configuration defined elsewhere.
  See section~\ref{sec:Config-Include} for a description.
  \item metaknobs, which incorporate predefined sets of configuration
  that are commonly used.
  See section~\ref{sec:Config-Metaknobs} for a description.
  \item a simple if/else syntax for conditional specification of 
  configuration. 
  See section~\ref{sec:Config-IfElse} for a description.
\end{itemize}
\Ticket{4325}
\Ticket{3894}
\Ticket{4319}
\Ticket{4031}
\Ticket{4211}

\item When hierarchical group quotas are used, and surplus
sharing is enabled, the quotas are now correctly computed
if slot weights are also enabled.
\Ticket{4324}

\item The default priority factor set for new users is now 1000.
This was changed from a default value of 1, because a value of 1 
leaves no room to boost the priority factor.
\Ticket{4282}

\item The \Condor{schedd} may now keep open a configurable number
of job event log files.
This improves performance over the previous behavior of
open, write, close done for each event.
New configuration variables \Macro{USERLOG\_FILE\_CACHE\_MAX} and 
\Macro{USERLOG\_FILE\_CACHE\_CLEAR\_INTERVAL} specify the number
of job event log files that may be kept open at the same time
and the periodic interval of time that passes
before the set of open files are closed.
\Ticket{4040}

\item The curl file transfer plug-in can now be used to transfer output
files in addition to input files.
\Ticket{4190}

\item New python bindings allow the user access to the same 
file locking protocol as HTCondor daemons.
\Ticket{4315}

\item The DAGMan node status file formatting has changed.
The format of the DAG node status file is now New ClassAds,
and the amount of information in the file has increased.
Section~\ref{sec:DAG-node-status} has details on node status files.
\Ticket{4115}

\item The new configuration variable \Macro{STARTER\_LOG\_NAME\_APPEND}
controls the file name extension of the log used by the \Condor{starter}.
\Ticket{4244}

\item The new configuration variable
 \Macro{ENVIRONMENT\_VALUE\_FOR\_UnAssigned<name>}
is intended for use with GPUs, where \texttt{<name>} is \texttt{GPUs}. 
It defines what GPU ID to assign to slots that have no assigned GPU.
Without this, the CUDA runtime would allow slots with no assigned GPU to use
all of the GPUs.
\Ticket{4320}

\item The batch system name \texttt{HTCondor} is now published in 
each job's environment.
\Ticket{4233}

\item New configuration variables \Macro{UDP\_NETWORK\_FRAGMENT\_SIZE} and
	\Macro{UDP\_LOOPBACK\_FRAGMENT\_SIZE} added to control UDP message
	fragmentation size over the network and loopback interface, 
	respectively.
\Ticket{4321}

\item The new \Condor{pool\_job\_report} tool for Linux platforms
composes and mails a report about all jobs run in the previous
24 hours on all execute machines within the pool. 
\Ticket{4267}

\item HTCondor now publishes more I/O statistics as job ClassAd attributes.
The new attributes are
\Attr{BlockReads},
\Attr{BlockWrites},
\Attr{RecentBlockReads},
\Attr{RecentBlockWrites},
\Attr{RecentBlockReadKbytes}, and
\Attr{RecentBlockWriteKbytes}.
\Ticket{3850}

\item The new job ClassAd attribute \Attr{SpoolOnEvict} facilitates
the debugging of failed jobs.
\Ticket{4292}

\item Memory corruption mitigation is enabled by additional linker flags,
when building HTCondor from source against system-shared
libraries installed by the distribution.
\Ticket{4153}

\item An experimental new feature to overlap the transfer of job output
with the execution of a subsequent job is documented with a link from
the HTCondor wiki page, 
\URL{https://htcondor-wiki.cs.wisc.edu/index.cgi/wiki?p=ExperimentalFeatures}.
\Ticket{4291}

\item An experimental new feature to provide custom output formatting
for \Condor{q} and \Condor{status} is documented with a link from
the HTCondor wiki page, 
\URL{https://htcondor-wiki.cs.wisc.edu/index.cgi/wiki?p=ExperimentalFeatures}.
\Ticket{4241}

\end{itemize}

\noindent Bugs Fixed:

\begin{itemize}

\item The \Condor{shared\_port} daemon no longer blocks
on a very unresponsive daemon.
\Ticket{4314}

\item vm universe jobs now report attribute \Attr{RemoteUserCPU} when
run on a KVM hypervisor.
CPU usage remains unreported by VMware hypervisors.
\Ticket{4337}

\item The \Condor{gridmanager} no longer assumes that a NorduGrid ARC job
with a reported exit code greater than 128 exited abnormally via a signal.
\Ticket{4342}

\item Many tools, including \Condor{off} and \Condor{restart} interpreted
the command line argument \Opt{-defrag} incorrectly as \Opt{-debug},
since both words start with the string \AdStr{de}.
The confusion has been fixed. 
Use of \Opt{-defrag} will now produce an error message, 
since it is not a valid option for these tools.
\Ticket{3717}

\item Fixed a crash by the \Condor{gpu\_discovery} tool,
when running on a 32-bit platform or on Windows and detecting via OpenCL.
\Ticket{4339}

\end{itemize}

%%%%%%%%%%%%%%%%%%%%%%%%%%%%%%%%%%%%%%%%%%%%%%%%%%%%%%%%%%%%%%%%%%%%%%
\subsection*{\label{sec:New-8-1-5}Version 8.1.5}
%%%%%%%%%%%%%%%%%%%%%%%%%%%%%%%%%%%%%%%%%%%%%%%%%%%%%%%%%%%%%%%%%%%%%%

\noindent Release Notes:

\begin{itemize}

\item HTCondor version 8.1.5 released on April 15, 2014.

\end{itemize}


\noindent New Features:

\begin{itemize}

\item The default configuration now implements a policy 
that disables preemption.
\Ticket{4281}

\item The protocol for interaction between \Condor{q} and the 
\Condor{schedd} daemon has been rewritten.
The new protocol does not require the \Condor{schedd} to fork a child process 
and does not cause blocking; 
the result is that the \Condor{schedd} should be able to handle
many concurrent \Condor{q} requests with minimal resource usage.
\Ticket{4111}

\item The specification in configuration for the size or amount of time
that a log file may grow has changed.
An explicit size or amount of time may still be specified for any
individual log file.
However, any log files not explicitly specified have a default maximum
size specified by the new configuration variable 
\Macro{MAX\_DEFAULT\_LOG}.
\Ticket{4246}

\item The new \Condor{urlfetch} tool is enables the  acquisition of
configuration with a query to a URL.
\Ticket{4018}

\item The \Prog{cream\_gahp} and \Prog{nordugrid\_gahp} can now talk to
servers over IPv6.
\Ticket{4243}

\item The python bindings can now accept a list of \Condor{collector} hosts
in the constructor of the \texttt{Collector} object.  
This eases use of the bindings for high availability setups.
\Ticket{4245}

\item The new python binding \texttt{transaction} creates a transaction
with the \Condor{schedd},
providing a way to submit multiple clusters of jobs
or edit multiple attributes atomically.
\Ticket{4225}

\item New configuration variable \Macro{NEGOTIATOR\_MAX\_TIME\_PER\_CYCLE}
places an upper time limit on the time spent in each negotiation cycle.
\Ticket{4271}

\item The configuration variable \Macro{VALID\_SPOOL\_FILES} has been redefined
to list only files that the system administrator determines must not
be removed by \Condor{preen}.
The new configuration variable \Macro{SYSTEM\_VALID\_SPOOL\_FILES} contains 
a predetermined list of files that are known to be valid at 
the time HTCondor was built. 
\Condor{preen} will use the union of these two configuration variables 
as the set of valid files that should not be removed from the \MacroNI{SPOOL}
directory.
\Ticket{4257}

\item The new configuration variable \Macro{OFFLINE\_MACHINE\_RESOURCE\_<name>}
is used to identify a custom machine resource as offline,
so that the resource will not be allocated to any slot.
\Ticket{4177}

\item The default value of configuration variable 
\Macro{NEGOTIATOR\_USE\_WEIGHTED\_DEMAND} has been changed from 
\Expr{False} to \Expr{True}.
\Ticket{4238}

\item The new configuration variable 
\Macro{NEGOTIATOR\_TRIM\_SHUTDOWN\_THRESHOLD} can be used to avoid 
making matches to resources that are about to go away. 
It is primarily of interest to glidein pools.  
Section~\ref{param:NegotiatorTrimShutdownThreshold} details the new
configuration variable.
\Ticket{4266}

\item No user-visible changes result from reductions in the quantity of
unused memory within DaemonCore data structures.
\Ticket{4206}

\item The \Condor{negotiator} logs more information about its round robin
iteration to ease debugging.
\Ticket{3871}

\item Some communications between daemons will cause fewer network timeouts,
as the reading of commands no longer blocks while
waiting for completion of the command.
\Ticket{4237}

\end{itemize}

\noindent Bugs Fixed:

\begin{itemize}

\item Fixed a bug that affected \Condor{on}, \Condor{off}, \Condor{restart},
\Condor{reconfig}, and \Condor{set\_shutdown}. 
When multiple machines were named on the command line, 
these tools could report 
\begin{verbatim}
Can't find address for master XXXX 
\end{verbatim}
for some daemons,
even though the daemons were properly advertised to the \Condor{collector}.
\Ticket{4207}

\item Fixed a bug that could have caused the \Condor{startd} to become 
unresponsive when starting a job obtained via the Work Fetch Hook.
\Ticket{4210}

\item Fixed a bug that could have caused the \Condor{schedd} to advertise a 
stale address in the \Attr{ScheddIpAddr} attribute of its submitter ClassAds,
resulting in other daemons being unable to contact it.
The problem occurred when using both the \Condor{shared\_port} daemon and CCB,
and the value of configuration variable \Macro{CCB\_ADDRESS} was changed.
\Ticket{4250}

\item Fixed a bug introduced earlier in the 8.1 developer series that 
could cause \Condor{submit} to crash when reading 
large submit description files.
\Ticket{4260}

\item Fixed a bug that prevented a configuration variable 
from referring to itself,
when the previous value was defined by the code,
rather than within a configuration file.
\Ticket{4256}

\item The temperature attributes output by the \Condor{gpu\_discovery} tool
contained values represented in Celsius, while
the names of these attributes ended in the letter 'F,' implying Fahrenheit.
The names of these attributes have been changed to end with the letter 'C.' 
For instance \Attr{<name>DieTempF} has been changed to \Attr{<name>DieTempC}.
\Ticket{4294}

\item The \Condor{startd} no longer generates this erroneous message
when a plugin can not be run:
\begin{verbatim}
Warning: Starter pid XXX is not associated with a claim.
A slot may fail to transition to Idle.
\end{verbatim}
\Ticket{4026}

\end{itemize}

%%%%%%%%%%%%%%%%%%%%%%%%%%%%%%%%%%%%%%%%%%%%%%%%%%%%%%%%%%%%%%%%%%%%%%
\subsection*{\label{sec:New-8-1-4}Version 8.1.4}
%%%%%%%%%%%%%%%%%%%%%%%%%%%%%%%%%%%%%%%%%%%%%%%%%%%%%%%%%%%%%%%%%%%%%%

\noindent Release Notes:

\begin{itemize}

\item HTCondor version 8.1.4 released on February 27, 2014.

\item This version of HTCondor includes all bug fixes from version 8.0.6,
as well as the new full port for the Red Hat Enterprise Linux 7.0 \emph{Beta} 
release on the x86\_64 architecture.
A full port includes support for the standard universe. 

\end{itemize}


\noindent New Features:

\begin{itemize}

\item When configured to use partitionable slots,
those slots running jobs can now be preempted by the 
\Condor{negotiator} daemon based on the value of 
the machine's configuration of \MacroNI{RANK}.
\Ticket{3667}

\item Improved support for publishing monitoring information about an
HTCondor pool to \TM{Ganglia}.
Added Ganglia statistics for total job starts and total job preemptions
within a \Condor{startd}.
This allows Ganglia to graph the total job preemptions
across all \Condor{startd} daemons in a pool.
See section~\ref{sec:Config-gangliad} for configuration variable definitions,
and section~\ref{sec:monitor-ganglia} for details about monitoring
with Ganglia.
\Ticket{4151}
\Ticket{3965}

\item The grid universe can now be used to create and manage VM instances
in Google Compute Engine (GCE), using the new grid type \SubmitCmdNI{gce}.
\Ticket{3833}

\item As a scalability improvement for Unix platforms, 
the \Condor{shared\_port} daemon no longer forks on incoming connections.
\Ticket{4094}

\item \Condor{ssh\_to\_job} and interactive jobs no longer try to 
connect to held jobs.
They instead report the hold and the reason why the job is being held.
\Ticket{3867}

\item Improved the restart time of the \Condor{schedd} after it has crashed.
\Ticket{4169}

\item The new configuration variable \Macro{EC2\_RESOURCE\_TIMEOUT} sets
the amount of time that HTCondor will wait for an unresponsive EC2 service 
before placing the corresponding jobs on hold.
\Ticket{4113}

\item The new python binding \Procedure{refreshGSIProxy}
can refresh a remote job's GSI proxy as a part of the \texttt{Schedd} object.
\Ticket{4116}

\item By default, 
the TCP keep alive interval is automatically tuned to 5 minutes.  
This causes at least one packet to be sent on established,
but idle, TCP connections once every 5 minutes, 
and it speeds up the detection of connections that were silently dropped 
by NAT or firewall devices.
Without this, 
the \Condor{shadow} may not reliably recover from transient network failures.
This behavior is controlled by the new configuration variable
\Macro{TCP\_KEEPALIVE\_INTERVAL}.
Setting this variable to 0 restores the prior behavior.
In addition, the configuration variable \Macro{CCB\_HEARTBEAT\_INTERVAL} 
default value has been reduced to 5 minutes.
\Ticket{4122}

\item New python \Code{ClassAd} module function calls 
\Procedure{Attribute}, \Procedure{Function}, \Procedure{Literal},
\Procedure{flatten}, \Procedure{matches}, and \Procedure{symmetricMatch}
aid the composition of ClassAd expressions.
It should now be possible to build expressions directly
in python, without having to resort to string manipulation.
\Ticket{4154}

\item For those that use the Python bindings,
the \Env{LD\_LIBRARY\_PATH} environment variable no longer needs to be set.
\Ticket{4128}

\item The Python bindings are now compatible with Python 3.
\Ticket{4146}


\item Setting configuration variable
\Macro{DAGMAN\_ALWAYS\_USE\_NODE\_LOG} to \Expr{False}
or using the corresponding \Opt{-dont\_use\_default\_node\_log} option
to \Condor{submit\_dag} is no longer recommended.
It is no longer recommended to have \Condor{dagman} read the log files 
specified in the node job submit description files.
\Ticket{4091}

\item Invoking \Condor{fetchlog} with the \Arg{STARTD\_HISTORY} argument
now fetches all \Condor{startd} history by concatenating all instances 
of log files resulting from rotation to the current history log.
\Ticket{4152}

\item Several general mechanisms for specifying user-defined \Condor{startd} 
resources have been enhanced,
so that GPUs can be easily defined and used.
New to this 8.1.4 version of HTCondor is the allocation of user defined
resources (especially GPUs) with partitionable and dynamic slots.
This includes having HTCondor automatically set the environment variable
\Env{CUDA\_VISIBLE\_DEVICES} for jobs that use CUDA GPUs
and \Env{GPU\_DEVICE\_ORDINAL} for jobs that use OpenCL GPUs.

The mechanism defines configuration variables 
\Macro{MACHINE\_RESOURCE\_<name>} and 
\Macro{MACHINE\_RESOURCE\_INVENTORY\_<name>}
to specify the definition user-defined resources with a list of resource
identifiers.  
When HTCondor allocates one of these user-defined resources to a slot, 
it will also publish this assignment within the slot's ClassAd 
using the new job ClassAd attribute \Attr{Assigned<name>}.
And, it will define in the job's environment the variable
\Env{\_CONDOR\_Assigned<name>}.
The new configuration variable \Macro{ENVIRONMENT\_FOR\_Assigned<name>}
also sets further environment variables.
\Ticket{4141}
\Ticket{4148}

\item The new \Condor{gpu\_discovery} tool detects CUDA and OpenCL GPUs,
reporting them in the format needed to configure GPU resources 
using the configuration variable
\Macro{MACHINE\_RESOURCE\_INVENTORY\_GPUs}.
\Ticket{3386}

\item Two new pre-defined configuration variables are referenced with
\MacroU{DETECTED\_PHYSICAL\_CPUS} and \MacroU{DETECTED\_CPUS}.
\MacroUNI{DETECTED\_PHYSICAL\_CPUS} contains the number of 
physical (non-hyperthreaded) CPUs. 
\MacroUNI{DETECTED\_CPUS} will match the value of
either \MacroNI{DETECTED\_CORES} or \MacroNI{DETECTED\_PHYSICAL\_CPUS}, 
depending on the state of \Macro{COUNT\_HYPERTHREAD\_CPUS}.
The default value of \Macro{NUM\_CPUS} now defaults to the value
of \MacroNI{DETECTED\_CPUS}.
\Ticket{4197}

\item \Condor{q} will now show the macro-expanded job description from the attribute
\Attr{MATCH\_EXP\_JobDescription} instead of \Attr{JobDescription} if it is available.
\Ticket{4110}

\end{itemize}

\noindent Bugs Fixed:

\begin{itemize}

\item Fixed a small memory leak that was triggered by failed
file transfer attempts.
\Ticket{4134}

\item Fixed a bug that would leak one socket in each daemon,
when \Expr{NO\_DNS = True}.
\Ticket{4140}

\item Changed the way the \Condor{startd} allocates CPUs to
slots in configurations where there are more slots than CPUs.
CPUs are now distributed equally between slots that are not configured
to receive a specific number 
(using configuration variable \Macro{SLOT\_TYPE\_<N>}).
Before this change, these slots received 1 CPU each.
The new behavior matches how other slot resources are distributed.
\Ticket{3249}

\item The failure to terminate an EC2 grid universe job instance,
because the instance no longer exists at the service, 
is now considered a successful termination.  
This allows EC2 grid universe jobs to exit the queue, 
if the service purges termination records quickly.
\Ticket{4133}

\item HTCondor now interacts with EC2 services by using \Code{POST}
instead of \Code{GET},
which permits more services to accept user data with size greater than 8Kbytes.
\Ticket{4004}

\item Improved the handling of the \SubmitCmd{coresize} 
submit description file command,
by allowing values larger than 4Gbytes.
\Ticket{4155}

\item Fixed a bug that caused job arguments to not be displayed in the
default output of \Condor{q} when the submit description file used the
new syntax for job arguments.
\Ticket{2875}

\item The \Condor{startd} daemon will no longer abort when it exhausts
the supply of user-defined resources such as GPUs 
while assigning automatic resource shares to slots.
\Ticket{4176}

\end{itemize}

%%%%%%%%%%%%%%%%%%%%%%%%%%%%%%%%%%%%%%%%%%%%%%%%%%%%%%%%%%%%%%%%%%%%%%
\subsection*{\label{sec:New-8-1-3}Version 8.1.3}
%%%%%%%%%%%%%%%%%%%%%%%%%%%%%%%%%%%%%%%%%%%%%%%%%%%%%%%%%%%%%%%%%%%%%%

\noindent Release Notes:

\begin{itemize}

\item HTCondor version 8.1.3 released on December 23, 2013.
This developer release contains all bug fixes from HTCondor version 8.0.5.

\end{itemize}


\noindent New Features:

\begin{itemize}

\item The parsing of configuration has changed with respect to how
line continuation characters and comments interact.
The line continuation character no longer takes precedence over the
comment character.
\Ticket{4027}

\index{SUBSYS\_SUPER\_ADDRESS\_FILE macro@\texttt{<SUBSYS>\_SUPER\_ADDRESS\_FILE} macro}
\index{configuration macro!\texttt{SUBSYS\_SUPER\_ADDRESS\_FILE}}
\item When the super user issues a command 
or when the new \Condor{sos} tool invokes another tool,
the command can be serviced with a higher priority. 
This should be useful when attempting to get information from an
overloaded daemon, in order to diagnose or fix a problem.
Commands directed at the \Condor{schedd} or \Condor{collector} daemons 
have this ability by default.
Other DaemonCore daemons require configuration using the new 
configuration variable
\MacroB{<SUBSYS>\_SUPER\_ADDRESS\_FILE}.
\Ticket{4029}

\item The dedicated scheduler cpu usage within the \Condor{schedd} is now
throttled, so that it cannot consume all of the cpu, while starving the vanilla
scheduler.  This throttle can be adjusted by the new configuration variable
\Macro{DEDICATED\_SCHEDULER\_DELAY\_FACTOR}.  
This variable, which defaults to five,
sets the ratio of time spent not in the dedicated scheduler to the 
time scheduling parallel jobs.  
With this default of five, 
a maximum of 20\% of the scheduler's time will go to scheduling
parallel jobs.
\Ticket{4048}

\item The new \Condor{defrag} daemon ClassAd attribute 
\Attr{MeanDrainedArrived}
measures the mean time between arrivals of fully drained machines,
and the new attribute \Attr{DrainedMachines} 
measures the total numbers of fully drained machines
which have arrived during the run time of this \Condor{defrag} daemon.
\Ticket{4055}

\item The new \Opt{-defrag} option for \Condor{status} queries ClassAds
of the \Condor{defrag} daemon.
\Ticket{4039}

\item Machine ClassAd attributes \Attr{ExpectedMachineQuickDrainingCompletion}
and \Attr{ExpectedMachineGracefulDrainingCompletion} are updated with their
completion times if there are no active claims,
making these attributes more useful in setting policy for
partitionable slots. 
\Ticket{3481}

\item In a DAG, the node retry number is now available as VARS macro
(see section~\ref{dagman:VARS}).
\Ticket{4032}

\item Macro substitution both within configuration and within submit
description files has been extended to specify and use  
an optional default value if a value is not defined.
Section~\ref{sec:Config-File-Macros} has details for configuration.
\Ticket{4033}

\item The Python bindings \Code{htcondor} module has 
a new \Procedure{read\_events} method to acquire an iterator of
an HTCondor event log file.
\Ticket{4071}

\item The new \Opt{-daemons} option to \Condor{who} prints information
about the HTCondor daemons running on the specified machine,
including the daemon's PID, IP address and command port.
\Ticket{4007}

\end{itemize}

\noindent Configuration Variable and ClassAd Attribute Additions and Changes:

\begin{itemize}

\item Configuration variable \Macro{DAGMAN\_DEFAULT\_NODE\_LOG}
has been made more powerful,
so that it can be defined in HTCondor configuration files, 
instead of being useful only when defined in a per-DAG configuration file.
See section~\ref{param:DAGManDefaultNodeLog} for details.
\Ticket{3930}

\item The new configuration variable \Macro{CORE\_FILE\_NAME} is used to set
the name that DaemonCore uses to create a core file,
in the event of a daemon crash.
The default value for this configuration variable appends the daemon name,
so a crash of the \Condor{schedd} would create a core file named
\File{core.SCHEDD}.
\Ticket{4100}

\item The new configuration variable \Macro{JOB\_EXECDIR\_PERMISSIONS}
defines the permissions on a job's scratch directory. 
It defaults to setting permissions as \emph{0700}.
\Ticket{4016}

\item The following recently added machine ClassAd attributes have been renamed.
\begin{description}
\item \Attr{TotalJobStarts} became \Attr{JobStarts}.
\item \Attr{RecentTotalJobStarts} became \Attr{RecentJobStarts}.
\item \Attr{TotalPreemptions} became \Attr{JobPreemptions}.
\item \Attr{RecentPreemptions} became \Attr{RecentJobPreemptions}.
\item \Attr{TotalRankPreemptions} became \Attr{JobRankPreemptions}.
\item \Attr{RecentTotalRankPreemptions} became \Attr{RecentJobRankPreemptions}.
\item \Attr{TotalUserPrioPreemptions} became \Attr{JobUserPrioPreemptions}.
\item \Attr{RecentTotalUserPrioPreemptions} became \Attr{RecentJobUserPrioPreemptions}.
\end{description}
\Ticket{4101}

\item The new \Condor{schedd} statistics ClassAd attribute
\Attr{Autoclusters} gives the number of active autoclusters.
\Ticket{4020}

\end{itemize}

\noindent Bugs Fixed:

\begin{itemize}

\item None.

\end{itemize}

\noindent Known Bugs:

\begin{itemize}

\item None.

\end{itemize}

\noindent Additions and Changes to the Manual:

\begin{itemize}

\item None.

\end{itemize}


%%%%%%%%%%%%%%%%%%%%%%%%%%%%%%%%%%%%%%%%%%%%%%%%%%%%%%%%%%%%%%%%%%%%%%
\subsection*{\label{sec:New-8-1-2}Version 8.1.2}
%%%%%%%%%%%%%%%%%%%%%%%%%%%%%%%%%%%%%%%%%%%%%%%%%%%%%%%%%%%%%%%%%%%%%%

\noindent Release Notes:

\begin{itemize}

\item HTCondor version 8.1.2 released on October 31, 2013.
This 8.1.2 release contains all bug fixes from HTCondor version 8.0.4.

\end{itemize}


\noindent New Features:

\begin{itemize}

\item \Condor{config\_val} now supports \Opt{-dump} and \Opt{-verbose}
options to query configuration remotely from daemons.
\Ticket{3894}

\item The \Condor{chirp} protocol and command line tool has been
enhanced to support lower-cost, delayed updates to the job
ClassAd residing in the \Condor{schedd}; updates occur as other communications
take place, eliminating the overhead of a separate update.
These two new Chirp commands,
\Opt{set\_job\_attr\_delayed} and \Opt{get\_job\_attr\_delayed} allow the job
to send lightweight notification for events such as progress
monitoring, which need not be durable.
\Ticket{3353}

\item \Condor{history} has been enhanced to support
remote history using new \Opt{-pool} and \Opt{-name} options.
\Ticket{3897}

\item Matchmaking in the \Condor{negotiator} may be made aware of resources
available for partitionable slots.
This permits multiple jobs to be matched against a partitionable slot
during a single negotiation cycle.
The new policies discussed in Section~\ref{sec:consumption-policy}
are set using new configuration variables and are known as consumption policies.
\Ticket{3435}

\item Definition syntax for the authorization configuration variables
\Macro{ALLOW\_*} and \Macro{DENY\_*} has been expanded to permit
the specification of Unix netgroups.
See section~\ref{sec:Security-Authorization} for the syntax.
\Ticket{3859}

\item Definition syntax for the configuration variable
\Macro{QUEUE\_SUPER\_USERS} has been expanded to accept a specification
of Unix user groups.
See section~\ref{param:QueueSuperUsers} for the syntax.
\Ticket{3859}

\item To ensure that a grid universe job running at an EC2 service
terminates, 
HTCondor now checks after a fixed time interval 
that the job actually has terminated,
instead of relying on the service's potentially unreliable 
job shut down indication.
If the job has not terminated after a total of four checks,
the job is placed on hold; it does not leave the queue marked as completed.
\Ticket{3438}

\item Email alerts about file transfers taking longer than
\Macro{MAX\_TRANSFER\_QUEUE\_AGE} are now grouped together
to reduce the number of email messages that are sent.

\item Floating point values in Old ClassAds are now printed in a more
human-readable format, while retaining 64-bit double precision.
In previous versions, these values were always printed in scientific
notation.
\Ticket{3928}

\item \Condor{ssh\_to\_job} now works with grid universe jobs
which use EC2 resources.
\Ticket{1548}

\item Machine ClassAd attributes \Attr{Disk} and \Attr{TotalDisk} 
are now published as 64-bit integers,
rather than being capped at the maximum value of a 32-bit integer.
\Ticket{1784}

\item In an effort to improve scalability under heavy load, the tuning
configuration variable \Macro{MAX\_REAPS\_PER\_CYCLE} is exposed,
as defined at section~\ref{param:MaxReapsPerCycle}.
The default for this variable changed from 1 to 0.
\Ticket{3992}

\item To reduce the overwhelming quantity of per-user \Condor{schedd} 
statistics that are generated when configuration variables 
\MacroNI{SCHEDD\_COLLECT\_STATS\_FOR\_<Name>} or 
\MacroNI{SCHEDD\_COLLECT\_STATS\_BY\_<Name>} are used, 
the statistics are now published at verbosity level 2,
instead of verbosity level 1.
\Ticket{3980}

\item The Python bindings now include the \Code{Negotiator} class to
manage users and their priorities.
\Ticket{3893}

\item The Python bindings now provide automatic conversions from 
dictionaries to ClassAds,
so they can accept a dictionary directly as an argument,
rather than constructing a ClassAd from the dictionary.
\Ticket{3892}

\item The Python bindings \Code{ClassAd} module has 
\Procedure{quote} and \Procedure{unquote} 
methods to help create string literals. 
\Ticket{3900}

\item The Python bindings \Code{ClassAd} module has new
methods \Procedure{parseAds} and \Procedure{parseOldAds} 
that implement an iterator over ClassAds, in the New ClassAd and 
Old ClassAd format. 
\Ticket{3918}

\item The ordering of adding attributes to the machine ClassAd has been
changed, such that the attributes \Attr{Draining}, \Attr{DrainingRequestId},
and \Attr{LastDrainStartTime} are now added before the job retirement
is calculated.
This allows a decision about preemption to be made based on if
a machine is currently draining. 
\Ticket{3901}

\end{itemize}

\noindent Bugs Fixed:

\begin{itemize}

\item When \Macro{USE\_PID\_NAMESPACES} is \Expr{True}, 
the soft kill signal is now successfully sent to the job.
Previously, a \Condor{rm}
command of such a job would not remove the job until the
killing timeout had expired.
\Ticket{3981}

\item If a standard universe job exited without producing any
checkpoints and no checkpoint server was used, 
two spurious error messages would be logged to the \File{SchedLog},
as it tried to remove the old checkpoint images from the
non-existent checkpoint server.  
These error messages are no longer logged.
\Ticket{3919}

\item When configuration variable \Macro{STARTER\_RLIMIT\_AS} is set 
to its default value of 0, it means that there is no limit.  
This value was logged as a limit of 0Mb, leading to confusion.
Now, no message is logged in this default case.
\Ticket{3914}

\item Improved how the \Condor{schedd} notifies the \Condor{shadow}
and \Condor{gridmanager} about modifications to job ClassAds made using
\Condor{qedit}.
\Ticket{3909}

\item Grid universe jobs now use the correct executable file when
\SubmitCmd{copy\_to\_spool} is set to \Expr{True}.
Previously, the executable file named in the submit description file 
would be copied to the remote server, 
rather than the copy of the executable file stored in the spool directory.
\Ticket{3589}

\item The example configuration provided within files 
\File{condor\_config.generic} and \File{condor\_config.generic.redhat} 
has been updated to fix an inadequate expression defining 
\MacroNI{NEGOTIATOR\_POST\_JOB\_RANK} when the \Condor{startd} is 
configured to not run benchmarks, as \Attr{Kflops} would not be defined.
\Ticket{3589}

\item Fixed a Python binding crash due to a segmentation fault,
when evaluating an expression tree with an undefined reference.
The fix allows the user to define the \Code{ClassAd} scope 
within which an expression tree is evaluated.
\Ticket{3910}

\item The Python bindings now include a correct conversion of
\Code{absTime} and \Code{relTime} ClassAd literals to the 
corresponding Python types.
\Ticket{3911}

\end{itemize}


%%%%%%%%%%%%%%%%%%%%%%%%%%%%%%%%%%%%%%%%%%%%%%%%%%%%%%%%%%%%%%%%%%%%%%
\subsection*{\label{sec:New-8-1-1}Version 8.1.1}
%%%%%%%%%%%%%%%%%%%%%%%%%%%%%%%%%%%%%%%%%%%%%%%%%%%%%%%%%%%%%%%%%%%%%%

\noindent Release Notes:

\begin{itemize}

\item HTCondor version 8.1.1 released on September 17, 2013.
This release contains all bug fixes from the stable release version 8.0.2.

\end{itemize}


\noindent New Features:

\begin{itemize}

\item Reduced the number of calls to the service when managing EC2 jobs. This
should increase the number of EC2 jobs HTCondor can manage on a given service
without overloading it.
\Ticket{3683}

\item When configuration variable \Macro{USE\_SHARED\_PORT} is \Expr{True},
\Macro{SHARED\_PORT} will now be automatically added to \Macro{DAEMON\_LIST}.
To disable this new behavior, set the new configuration variable:
\begin{verbatim}
  AUTO_INSERT_SHARED_PORT_IN_DAEMON_LIST = False
\end{verbatim}
\Ticket{3799}

\item Floating point values in ClassAds are now printed as 64-bit
double precision values when sent over the network, written to disk, and
displayed using the \Opt{-long} or \Opt{-autoformat} options of
\Condor{status} and \Condor{q}.
\Ticket{3363}

\item In the Pegasus/DAGMan workflow metrics,
as documented in section ~\ref{sec:DAGMetrics},
the two metrics
\Expr{dagman\_id} and \Expr{parent\_dagman\_id} are now reported
as the \Attr{ClusterId} of the \Condor{dagman} job.  This
eliminates any privacy concerns with reporting the \Condor{schedd} 
daemon's address,
which includes the submit machine's IP address.

\item The python bindings now can perform the equivalent of 
\Condor{ping} as a part of the \texttt{SecMan} object.
\Ticket{3857}

\item The \Condor{gridmanager} and \Condor{ft-gahp} now create
dynamic security session for performing file transfers.
Previously, the security configuration had to be set in a special
way for file transfers with the \Condor{ft-gahp} to work.
\Ticket{3536}

\end{itemize}

\noindent Configuration Variable and ClassAd Attribute Additions and Changes:

\begin{itemize}

\item The new configuration variable \Macro{USE\_RESOURCE\_REQUEST\_COUNTS}
is a boolean value that defaults to \Expr{True}, 
reducing the latency of negotiation 
when there are many jobs next to each other in the queue 
with the same auto cluster, and many matches are being made.
\Ticket{3585}

\item Four new machine ClassAd attributes are advertised.
\Attr{TotalJobStarts} is the total number of jobs started by 
this \Condor{startd} daemon since it booted. 
\Attr{RecentTotalJobStarts} is the number of jobs started in the
last twenty minutes.  
Similarly, \Attr{TotalPreemptions} is
the number of jobs preempted since the \Condor{startd} daemon started,
and \Attr{RecentTotalPreemptions} is 
the number in the last 20 minutes.
\Ticket{3712}

\item \Macro{FILE\_TRANSFER\_DISK\_LOAD\_THROTTLE} now accepts tabs in addition to spaces as delimiters.
\Ticket{3798}

\item Configuration variable \Macro{VALID\_SPOOL\_FILES} has been expanded
to accept a single asterisk wild card character in each listed file name.
\Ticket{3764}

\item The new configuration variable \Macro{GAHP\_DEBUG\_HIDE\_SENSITIVE\_DATA}
is a boolean value that defaults to hiding sensitive data 
such as security keys and passwords
when communication with a GAHP server is written to a daemon log.
\Ticket{3536}

\item The default value of configuration variable 
\Macro{ENABLE\_CLASSAD\_CACHING} has changed to \Expr{True} for all
daemons other than the \Condor{shadow}, \Condor{starter}, and \Condor{master}.
\Ticket{3441}

\end{itemize}

\noindent Bugs Fixed:

\begin{itemize}

\item The \Condor{gridmanager} now does proper failure recovery when
submitting EC2 grid universe jobs to services that do not support 
the EC2 ClientToken parameter.
Previously, if there was a failure when submitting jobs to OpenStack
or Eucalyptus, the jobs could be submitted twice.
\Ticket{3682}

\item Fixed the printing of nested ClassAds, so that the nested ClassAds
can be read back properly.
\Ticket{3772}

\item Fixed a bug between the \Condor{gridmanager} and \Condor{ft-gahp}
that caused file transfers to fail if one of the two daemons was older
than version 8.1.0.
\Ticket{3856}

\item Fixed a bug that caused substitution in configuration variable
evaluation to ignore per-daemon overrides. 
This is a long standing bug that may result in subtle changes
to the way your configuration files are processed.
An example of how substitution works with the per-daemon overrides
is in section \ref{sec:Config-File-Macros}.
\Ticket{3822}

\item Fixed a bug that caused the command
\begin{verbatim}
  condor_submit -
\end{verbatim}
to be interpreted as an interactive submit,
rather than a request to read input from \File{stdin}.
\Condor{qsub} was also modified to be immune to this bug,
such that it will still work with other versions of HTCondor containing
the bug.
\Ticket{3902}

\end{itemize}

\noindent Known Bugs:

\begin{itemize}
\item DAGMan recovery mode does not work for Pegasus-generated sub-DAGs.
For sub-DAGs, doing \Condor{hold} or \Condor{release} on
the \Condor{dagman} job, or stopping and re-starting the
\Condor{schedd} with the DAGMan
job in the queue will result in failure of the DAG.  This can be
avoided by doing a \Condor{rm} of the DAGMan job, which produces a Rescue
DAG, and re-submitting the DAG; the Rescue DAG is automatically run.
This bug was introduced in HTCondor version 8.0.1, and it also appears
in versions 8.0.2, 8.1.0, and 8.1.1.
\Ticket{3882}

\end{itemize}

\noindent Additions and Changes to the Manual:

\begin{itemize}

\item None.

\end{itemize}


%%%%%%%%%%%%%%%%%%%%%%%%%%%%%%%%%%%%%%%%%%%%%%%%%%%%%%%%%%%%%%%%%%%%%%
\subsection*{\label{sec:New-8-1-0}Version 8.1.0}
%%%%%%%%%%%%%%%%%%%%%%%%%%%%%%%%%%%%%%%%%%%%%%%%%%%%%%%%%%%%%%%%%%%%%%

\noindent Release Notes:

\begin{itemize}

\item HTCondor version 8.1.0 released on August 5, 2013.
This release contains all bug fixes from the stable release version 8.0.1.

\end{itemize}


\noindent New Features:

\begin{itemize}

\item Added support for publishing information about an HTCondor pool 
to \TM{Ganglia}.
See section~\ref{sec:Config-gangliad} on 
page~\pageref{sec:Config-gangliad} for configuration variable details.
\Ticket{3515}

\item Improved the performance of the \Condor{collector} daemon when running
at sites that do not observe daylight savings time.
\Ticket{2898}

\item \Condor{q}, \Condor{rm}, \Condor{status} and \Condor{qedit} are now 
more consistent in the way they handle the \Opt{-constraint} option.
\Ticket{1156}

\item The new \Condor{dagman\_metrics\_reporter} executable
with manual page at ~\pageref{man-condor-dagman-metrics-reporter},
reports metrics for DAGMan workflows running under Pegasus.  \Condor{dagman}
now generates an output file of the relevant metrics,
as described at ~\pageref{sec:DAGMetrics}.
\Ticket{3532}

\end{itemize}

\noindent Configuration Variable and ClassAd Attribute Additions and Changes:

\begin{itemize}

\item The default value of configuration variable
\Macro{COLLECTOR\_MAX\_FILE\_DESCRIPTORS} has changed to 10240,
and the default value of configuration variable 
\Macro{SCHEDD\_MAX\_FILE\_DESCRIPTORS} has changed to 4096.
This increases the scalability of the default configuration.
\Ticket{3626}

\item The new configuration variable
\Macro{FILE\_TRANSFER\_DISK\_LOAD\_THROTTLE} enables dynamic
adjustment of the level of file transfer concurrency in order to
keep the disk load generated by transfers below a specified level.
Supporting this new feature are configuration variables
\Macro{FILE\_TRANSFER\_DISK\_LOAD\_THROTTLE\_WAIT\_BETWEEN\_INCREMENTS},
\Macro{FILE\_TRANSFER\_DISK\_LOAD\_THROTTLE\_SHORT\_HORIZON}, and
\Macro{FILE\_TRANSFER\_DISK\_LOAD\_THROTTLE\_LONG\_HORIZON}.
\Ticket{3613}

\item The following new \Condor{schedd} ClassAd attributes are for
monitoring file transfer activity:
\AdAttr{TransferQueueMBWaitingToDownload},
\AdAttr{TransferQueueMBWaitingToUpload},
\AdAttr{FileTransferDiskThrottleLevel},
\AdAttr{FileTransferDiskThrottleHigh}, and
\AdAttr{FileTransferDiskThrottleLow}.
\Ticket{3613}

\item The default value for the configuration variable
\Macro{PASSWD\_CACHE\_REFRESH} has been changed from 300 seconds to
72000 seconds (20 hours).
\Ticket{3723}

\item The new configuration variables
\Macro{DAGMAN\_PEGASUS\_REPORT\_METRICS} and
\Macro{DAGMAN\_PEGASUS\_REPORT\_TIMEOUT}
set defaults used by the new \Condor{dagman\_metrics\_reporter} executable,
which reports metrics for DAGMan jobs running under Pegasus.
\Ticket{3532}

\end{itemize}

\noindent Bugs Fixed:

\begin{itemize}

\item HTCondor version 8.0.0 had an unintended change in the Chirp 
wire protocol.
This change caused \Condor{chirp} with the \Opt{put} option
to fail when the execute node
was running HTCondor version 7.8.x or earlier versions. 
HTCondor 8.0.1 and later
versions will now send the original wire protocol, and accept either the
original protocol, or the variant that HTCondor version 8.0.0 sends.
\Ticket{3735}

\item Fixed a bug that could cause the daemons to crash on Unix platforms,
if the operating system reported that a job owner's account 
did not exist, for example due to a temporary NIS or LDAP failure.
\Ticket{3723}

\item Fixed a bug that resulted in a misleading error message when
\Condor{status} with the \Opt{-constraint} option specified a constraint 
that could not be parsed.
\Ticket{1319}

\item Fixed a typo in the output of \Condor{q},
where a period was erroneously present within a heading.
\Ticket{3703}

\end{itemize}

\noindent Known Bugs:

\begin{itemize}

\item None.

\end{itemize}

\noindent Additions and Changes to the Manual:

\begin{itemize}

\item None.

\end{itemize}



%%%      PLEASE RUN A SPELL CHECKER BEFORE COMMITTING YOUR CHANGES!
%%%      PLEASE RUN A SPELL CHECKER BEFORE COMMITTING YOUR CHANGES!
%%%      PLEASE RUN A SPELL CHECKER BEFORE COMMITTING YOUR CHANGES!
%%%      PLEASE RUN A SPELL CHECKER BEFORE COMMITTING YOUR CHANGES!
%%%      PLEASE RUN A SPELL CHECKER BEFORE COMMITTING YOUR CHANGES!

%%%%%%%%%%%%%%%%%%%%%%%%%%%%%%%%%%%%%%%%%%%%%%%%%%%%%%%%%%%%%%%%%%%%%%
\section{\label{sec:History-8-0}Stable Release Series 8.0}
%%%%%%%%%%%%%%%%%%%%%%%%%%%%%%%%%%%%%%%%%%%%%%%%%%%%%%%%%%%%%%%%%%%%%%

This is a stable release series of HTCondor.
As usual, only bug fixes (and potentially, ports to new platforms)
will be provided in future 8.0.x releases.
New features will be added in the 8.1.x development series.

The details of each version are described below.

%%%%%%%%%%%%%%%%%%%%%%%%%%%%%%%%%%%%%%%%%%%%%%%%%%%%%%%%%%%%%%%%%%%%%%
\subsection*{\label{sec:New-8-0-7}Version 8.0.7}
%%%%%%%%%%%%%%%%%%%%%%%%%%%%%%%%%%%%%%%%%%%%%%%%%%%%%%%%%%%%%%%%%%%%%%

\noindent Release Notes:

\begin{itemize}

\item HTCondor version 8.0.7 not yet released.
%\item HTCondor version 8.0.7 released on Month Date, 2013.

\end{itemize}


\noindent New Features:

\begin{itemize}

\item The new configuration variable \Macro{JOB\_EXECDIR\_PERMISSIONS}
defines the permissions on a job's scratch directory. 
It defaults to setting permissions as \emph{0700}.
\Ticket{4208}


\end{itemize}

\noindent Bugs Fixed:

\begin{itemize}

\item Fixed the parsing of the \Condor{q} command line,
such that it looks for the entire option \Opt{-stream-results},
instead of only the portion \Opt{-stream}.
The code now matches what was previously documented in the manual page
and in the usage output.
\Ticket{4205}

\item Fixed a bug in the grid universe that could allow a removed job to
leave the queue without cleaning up all job state on the remote server.
\Ticket{4216}

\item Fixed the printing of nested ClassAds, so that the nested ClassAds
can be read back properly.
\Ticket{3772}

\item Fixed a bug in the \Condor{starter} that could cause a crash if
the job is evicted while a \Condor{ssh\_to\_job} command is active.
\Ticket{4251}

\end{itemize}

%%%%%%%%%%%%%%%%%%%%%%%%%%%%%%%%%%%%%%%%%%%%%%%%%%%%%%%%%%%%%%%%%%%%%%
\subsection*{\label{sec:New-8-0-6}Version 8.0.6}
%%%%%%%%%%%%%%%%%%%%%%%%%%%%%%%%%%%%%%%%%%%%%%%%%%%%%%%%%%%%%%%%%%%%%%

\noindent Release Notes:

\begin{itemize}

\item HTCondor version 8.0.6 released on February 11, 2014.

\end{itemize}


\noindent New Features:

\begin{itemize}

\item A full port, which includes support for the standard universe,
is available for the Red Hat Enterprise Linux 7.0 \emph{Beta} release
on the x86\_64 architecture.
\Ticket{3629}

\item HTCondor now forces proxies that it delegates 
to be a minimum of 1024 bits.
\Ticket{4168}

\end{itemize}

\noindent Bugs Fixed:

\begin{itemize}

\item Fixed a rare bug in the \Condor{negotiator} where 
completely disabled preemption achieved with
\Expr{PREEMPTION\_REQUIREMENTS = False}
might still cause a newly started job to be preempted.  
This bug was much more likely when
configuration variable \Macro{NEGOTIATOR\_CYCLE\_DELAY} was set lower 
than the default value.
\Ticket{4185}

\item Fixed a bug in the \Condor{schedd} which would cause it
to crash when running remotely submitted parallel universe jobs.
\Ticket{4163}

\item Fixed a bug with concurrency limits and partitionable slots,
in which the partitionable slot would consume a concurrency token,
even though the slot never runs jobs.  This would cause fewer jobs
than desired to run.
\Ticket{4145}

\item Fixed a bug in which the Job Exit work fetch hook was prematurely killed.
\Ticket{3669}

\item When the Windows installer was told to use VMware,
it configured a requirement for a 
\File{condor\_vmware\_local\_settings} file that it did not provide.
This caused the \Condor{vm-gahp} to fail to start,
unless the user created this file.
This configuration has been removed, so that the file is no longer required. 
\Ticket{4109}

\item Fixed a crash of the \Condor{shadow}, triggered when a disconnect
from the \Condor{starter} occurs just as the job terminates.
\Ticket{4127}

\item Modified the output of \Condor{q},
such that when invoked with options \Opt{-userlog} and \Opt{-af},
a blank line and a totals summary line are not displayed.
\Ticket{4045}

\item For hierarchical group quotas, 
fixed an incorrect count of the number of jobs when applying a quota.
\Ticket{4117}

\item Fixed the HTCondor module \Procedure{send\_command} python binding;
an incorrect argument was only the first character of the daemon's name,
instead of the full name. 
This affected the ability to turn off specific daemons,
as sent to the \Condor{master}.
\Ticket{4160}

\item Fixed the transfer of files larger than 4Gbytes,
such that they no longer stop before the transfer is completed.
This bug presented itself on all Windows systems and on all
platforms running on a 32-bit architecture.
\Ticket{4150}

\item Fixed a bug that caused \Condor{submit\_dag} to crash on very
large DAG input files, such as those larger than 2 Gbytes.
The new configuration variable \MacroNI{DAGMAN\_USE\_OLD\_DAG\_READER},
as detailed in section~\ref{param:DAGmanUseOldDagReader},
allows disabling the new file reader code put in place to fix the bug.
\Ticket{4171}

\item Fixed a bug that caused attempts to set the CPU affinity 
on Windows platforms to be quietly ignored.
\Ticket{4131}

\end{itemize}

%%%%%%%%%%%%%%%%%%%%%%%%%%%%%%%%%%%%%%%%%%%%%%%%%%%%%%%%%%%%%%%%%%%%%%
\subsection*{\label{sec:New-8-0-5}Version 8.0.5}
%%%%%%%%%%%%%%%%%%%%%%%%%%%%%%%%%%%%%%%%%%%%%%%%%%%%%%%%%%%%%%%%%%%%%%

\noindent Release Notes:

\begin{itemize}

\item HTCondor version 8.0.5 released on December 12, 2013.

\item Helper script \Condor{ckpt\_probe} has been missing
from HTCondor releases since version 7.8.1, and it is once again 
in this release.
As a result, the machine ClassAd attribute \Attr{CheckpointPlatform}
will change for standard universe jobs upon upgrade to HTCondor version 8.0.5.
This will prevent standard universe jobs started before the upgrade
from continuing, because the attribute change eliminates a match 
with upgraded machines. 
To work around this issue,
change the \Attr{LastCheckpointPlatform} attribute to be current,
such that all jobs that have produced a checkpoint will qualify to 
continue on the upgraded machines.
Make the change by using \Condor{qedit}. 
For example, use
\footnotesize
\begin{verbatim}
condor_qedit -constraint 'LastCheckpointPlatform == "LINUX INTEL 2.6.x normal N/A"'
    LastCheckpointPlatform "LINUX INTEL 2.6.x normal N/A ssse3 sse4_1 sse4_2"
\end{verbatim}
\normalsize
where \Attr{CheckpointPlatform} after upgrade shows as
\Expr{LINUX INTEL 2.6.x normal N/A ssse3 sse4\_1 sse4\_2}.
\Ticket{4025}

\end{itemize}


\noindent New Features:

\begin{itemize}

\item None.

\end{itemize}

\noindent Bugs Fixed:

\begin{itemize}

\item Fixed a bug that resulted in incorrect values for ClassAd attributes
\Attr{RemoteGroupQuota} and \Attr{SubmitterGroupQuota}.
These attributes are commonly used in an expression that defines
\Macro{PREEMPTION\_REQUIREMENTS}.
\Ticket{4093}

\item Fixed a permissions bug that prevented Java universe jobs
from running.
\Ticket{4087}

\item Fixed a bug in which job output files may not have been properly truncated
at start up on Windows platforms. 
\Ticket{4097}

\item Fixed bug with \Condor{submit} 
when invoked with the \Opt{-interactive} option and cgroups where enabled.
The shell was terminated immediately after it started.
\Ticket{4028}

\item Prevent illegal values from being written into the job queue
or accounting recovery logs, in order to minimize the chance of errors when
the \Condor{schedd} or \Condor{negotiator} are restarted. Also
have \Condor{qedit} validate attribute names and values.
\Ticket{3616}

\item The \Condor{starter} now writes to the log file
\File{StarterLog.slotX} when running work fetch jobs, mirroring the
behavior when running jobs that come from a \Condor{schedd}.
Previously, log file \File{StarterLog} was used for all work fetch
jobs.
\Ticket{3091}

\item Fixed a Python binding crash due to a segmentation fault, when evaluating an expression tree with an undefined reference.
\Ticket{3910}

\item Fixed the cURL file transfer plugin such that it now works
on Windows platforms.
\Ticket{3979}
\Ticket{4012}

\item Fixed a bug that caused cream grid universe jobs to fail, 
if the submit description file contained 
submit commands of \SubmitCmd{environment} or \SubmitCmd{cream\_attributes}.
\Ticket{4037}

\item Fixed a bug that could cause the \Condor{schedd} to crash when
starting a local universe job.
\Ticket{4088}

\item Fixed a bug that caused \File{stdout} and \File{stderr} of
nordugrid grid universe jobs to be lost when the remote 
NorduGrid ARC server was using HTCondor as its local batch scheduler.
\Ticket{4017}

\item \Condor{status} with the \Opt{-t} option now consistently 
specifies the \Opt{-total} option. 
The \Opt{-target} option will now be distinguished, as it requires
at least \Opt{-targ} in its specification.
\Ticket{4096}

\item Restored the omitted HTCondor Perl module.
\Ticket{4098}

\item For RPM installations, the post-install script now
checks to make sure that the
SELinux \Code{unconfined\_execmem\_exec\_t type} exists before trying to
add it to the \File{/usr/sbin/condor\_startd} file context.
\Ticket{4034}

\item Fixed a bug in which both daemons and tools may have crashed due to
the inability to resolve a host name defined by 
configuration variable \Macro{COLLECTOR\_HOST} to an IP address.
\Ticket{3946}

\item Changed the algorithm for calculating the value of attribute
\Attr{RecentDaemonCoreDutyCycle} such
that the published value will never be negative.
\Ticket{4052}

\item For VMware vm universe jobs,
the \Condor{vm-gahp} removed from the vm description file any cdrom, 
floppy drive, serial or parallel devices.
Now, only devices that do not refer to image files are removed,
as the devices may be useful to the virtual machine.
\Ticket{4002}

\end{itemize}


%%%%%%%%%%%%%%%%%%%%%%%%%%%%%%%%%%%%%%%%%%%%%%%%%%%%%%%%%%%%%%%%%%%%%%
\subsection*{\label{sec:New-8-0-4}Version 8.0.4}
%%%%%%%%%%%%%%%%%%%%%%%%%%%%%%%%%%%%%%%%%%%%%%%%%%%%%%%%%%%%%%%%%%%%%%

\noindent Release Notes:

\begin{itemize}

\item HTCondor version 8.0.4 released on October 24, 2013.

\item A clipped version of HTCondor is now provided for Ubuntu 12.04 
on the x86\_64 architecture.
\Ticket{3972}

\end{itemize}


\noindent New Features:

\begin{itemize}

\item None.

\end{itemize}

\noindent Configuration Variable and ClassAd Attribute Additions and Changes:

\begin{itemize}

\item The new configuration variable \Macro{DYNAMIC\_RUN\_ACCOUNT\_LOCAL\_GROUP}
permits an administrator of a Windows machine to specify a local group other 
than the default \Expr{Users} for the \Expr{condor-reuse-slot<X>} run account.
\Ticket{3998}

\end{itemize}

\noindent Bugs Fixed:

\begin{itemize}

\item Added a retry to work around problems with slow
DHCP servers.  The result of the problem was that daemons
would not be able to determine their own host names,
and, among other problems, machine ClassAds would appear
in the output of \Condor{status} with blank host names.
\Ticket{3956}

\item A spurious warning in the log of the \Condor{startd} 
about undefined \MacroNI{PREEMPT\_VANILLA} expressions
is no longer generated.
\Ticket{3984}

\item EC2 grid universe jobs may provide data in files.
File contents of null-terminated strings worked correctly,
but binary data did not.
This bug fix makes sure that both kinds of data are read correctly.
\Ticket{3924}

\item EC2 grid universe jobs on OpenNebula work better for user data
specified with \SubmitCmd{ec2\_user\_data},
where the size of the data is larger than 2Kbytes.
\Ticket{3923}

\item Fixed a bug preventing HTTP file transfers from following redirects.
HTTP file transfers also now fail if the file does not exist,
but the server does exist.
\Ticket{3904}

\item Fixed a rare bug in which the \Condor{schedd} would exit
with an ERROR message pertaining to \Procedure{select},
when running with a very heavy load and many \Condor{shadow} daemons
time out.
\Ticket{3947}

\item Fixed a bug in \Condor{ssh\_to\_job}
that caused it to kill the job when the ssh session exited,
if cgroups were enabled.
\Ticket{3921}

\item If a standard universe job exited without producing any checkpoints 
and no checkpoint server was used, 
two spurious error messages would be logged to the \File{SchedLog}, 
as it tried to remove the old checkpoint images from the 
non-existent checkpoint server. 
These error messages are no longer logged.
\Ticket{3919}

\item The configuration file for Windows erroneously had two entries
for configuration variable \Macro{JAVA\_CLASSPATH\_SEPARATOR},
with the second entry specifying the
separator used on Unix platforms, which overrode the first entry. 
The Unix separator no longer affects this configuration file
for Windows platforms.
\Ticket{3957}

\item The NorduGrid GAHP no longer consumes all of the CPU when run
with threaded Globus libraries.
\Ticket{3958}

\item Fixed a bug that could cause ClassAd function \Procedure{quantize} to not
evaluate properly when its second argument was a list and ClassAd caching
was enabled.
\Ticket{3967}

\item Fixed a bug that caused the configuration variable setting 
\Expr{STARTD\_CRON\_AUTOPUBLISH = If\_Changed} to not work correctly.  
Updates were incorrectly sent to the \Condor{collector}
in many cases when no attribute value had changed.
\Ticket{3983}

\item Fixed a bug that could cause \Condor{q} \Opt{-analyze} or 
\Opt{-better-analyze}
to sometimes crash on an Illegal Instruction, 
when the \Attr{Requirements} expression contained a function.
\Ticket{3985}

\item The configuration variables \Macro{UPDATE\_COLLECTOR\_WITH\_TCP}
and \Macro{TCP\_UPDATE\_COLLECTORS} are now respected when forwarding
ClassAds to an HTCondorView server and setting \Macro{CONDOR\_VIEW\_HOST}.
\Ticket{3986}

\item \Condor{who} now prints an error message when passed an invalid argument.
\Ticket{3987}

\item Changed the script that finds the VMware tools to use the standard
environment variables to find \File{Program Files}, instead of using a
hard coded path.
This fixes installation for both 64-bit and non-English language Windows 
platforms.
\Ticket{272}

\end{itemize}

\noindent Known Bugs:

\begin{itemize}

\item None.

\end{itemize}

\noindent Additions and Changes to the Manual:

\begin{itemize}

\item None.

\end{itemize}


%%%%%%%%%%%%%%%%%%%%%%%%%%%%%%%%%%%%%%%%%%%%%%%%%%%%%%%%%%%%%%%%%%%%%%
\subsection*{\label{sec:New-8-0-3}Version 8.0.3}
%%%%%%%%%%%%%%%%%%%%%%%%%%%%%%%%%%%%%%%%%%%%%%%%%%%%%%%%%%%%%%%%%%%%%%

\noindent Release Notes:

\begin{itemize}

\item HTCondor version 8.0.3 released on September 23, 2013.

\end{itemize}


\noindent New Features:

\begin{itemize}

\item None.

\end{itemize}

\noindent Configuration Variable and ClassAd Attribute Additions and Changes:

\begin{itemize}

\item None.

\end{itemize}

\noindent Bugs Fixed:

\begin{itemize}

\item When expressions for \Macro{SUSPEND},
\Macro{CONTINUE}, \Macro{PREEMPT}, and \Macro{KILL}
were evaluated by the \Condor{startd},
a resulting value of
\Expr{UNDEFINED} or \Expr{ERROR} caused an exception in the \Condor{startd}.
These values are now treated as \Expr{False},
eliminating the exception.
This fix addresses CVE-2013-4255.
\Ticket{3869}

\item Fixed a bug that prevented the use of simple host names to identify
machines when using the command-line tools. Before the fix,
\Expr{condor\_status bar.foo.org} would show machine ClassAds, but
\Expr{condor\_status bar} would not.
Both now show machine ClassAds.
\Ticket{3694}

\item Fixed a bug with \Condor{ssh\_to\_job} that occurs when
\Macro{USE\_PID\_NAMESPACES} is enabled.
The bug presented itself as \Condor{ssh\_to\_job}
running in a private pid namespace, 
and the user running \Condor{ssh\_to\_job} not being able to see 
their job with \Prog{ps} or \Prog{gdb}.
As a fix, \Condor{ssh\_to\_job} now runs in the global namespace, 
so that it can see the processes in the user's job.
\Ticket{3872}

\item Fixed a performance problem with the \Condor{qedit} command 
that would cause the \Condor{schedd} to run very slowly when 
\Condor{qedit} is run on a large number of jobs.  
\Condor{qedit} no longer writes an event to the job event log. 
\Ticket{3827}

\item Fixed a problem where the \Code{classad} python module would return
incorrect results when ClassAd caching is enabled.
\Ticket{3879}

\item DAGMan's updating of its job ClassAd with DAG status attributes no
longer causes extra events to be written to the job event log and event log.
\Ticket{3863}

\item Fixed a bug that caused the command
\begin{verbatim}
  condor_submit -
\end{verbatim}
to be interpreted as an interactive submit,
rather than a request to read input from \File{stdin}.
\Condor{qsub} was also modified to be immune to this bug,
such that it will still work with other versions of HTCondor containing
the bug.
\Ticket{3902}

\item Value 032 of the job ClassAd attribute \Attr{HoldReasonCode}
was being used for two different reasons.
Now, value 032 identifies that the maximum total input file 
transfer size was exceeded. 
Value 034 identifies that memory usage exceeds a memory limit.
\Ticket{3858}

\item The values of the job ClassAd attributes \Attr{RemoteSysCpu} 
and \Attr{RemoteUserCpu} are sometimes impossibly large.
This bug rarely occurs and is not well understood.
Code changes attempt to fix this problem.
\Ticket{3814}

\item Fixed a bug that caused DAG recovery mode to fail on
Pegasus-generated sub-DAGs.  (Recovery mode is invoked, for example,
when a \Condor{dagman} is held and released, or when a schedd is
restarted, as happens on a version upgrade.)
\Ticket{3882}

\end{itemize}

\noindent Known Bugs:

\begin{itemize}

\item None.

\end{itemize}

\noindent Additions and Changes to the Manual:

\begin{itemize}

\item None.

\end{itemize}


%%%%%%%%%%%%%%%%%%%%%%%%%%%%%%%%%%%%%%%%%%%%%%%%%%%%%%%%%%%%%%%%%%%%%%
\subsection*{\label{sec:New-8-0-2}Version 8.0.2}
%%%%%%%%%%%%%%%%%%%%%%%%%%%%%%%%%%%%%%%%%%%%%%%%%%%%%%%%%%%%%%%%%%%%%%

\noindent Release Notes:

\begin{itemize}

%\item HTCondor version 8.0.2 not yet released.
\item HTCondor version 8.0.2 released on August 22, 2013.

\item Debian 5 is past its end of life. 
Starting with this release, we no longer provide native packages or
tarballs for Debian 5.
\Ticket{3852}

\end{itemize}


\noindent New Features:

\begin{itemize}

\item None.

\end{itemize}

\noindent Configuration Variable and ClassAd Attribute Additions and Changes:

\begin{itemize}

\item The default value of \Macro{ENABLE\_DEPRECATION\_WARNINGS} 
has been changed to \Expr{False}.
\Ticket{3848}

\end{itemize}

\noindent Bugs Fixed:

\begin{itemize}

\item Implemented a workaround to avoid triggering a Linux kernel defect 
when using cgroups and suspending jobs.
\Ticket{3847}

\item Fixed a python bindings problem of missing converters by providing
\Code{pyclassad} as a shared library.
\Ticket{3780}

\item Fixed a file permission bug introduced in HTCondor version 7.9.2 that
prevented vm universe jobs from working when using the Xen or KVM
hypervisor.
\Ticket{3781}

\item Fixed a bug that could cause the \Condor{collector} to 
become unresponsive if the remote HTCondorView server, 
specified with configuration variable \Macro{CONDOR\_VIEW\_HOST},
becomes unavailable.
\Ticket{3758}

\item Code to publish Linux distribution attributes in the machine ClassAd
is now more robust in the event that the \File{/etc/issue} file was edited.
\Ticket{3854}

\item Fixed a bug that could cause jobs to be incorrectly placed on hold upon
	completion with a hold reason claiming an out-of-memory event.
\Ticket{3824}

\item Fixed a bug that prevented work fetch scripts from running
on systems where cgroup based tracking and management was enabled.
\Ticket{3819}

\item Fixed a bug that could cause the \Condor{negotiator} to give out the same
slot twice, and result in a scary entry in the \File{NegotiatorLog} file 
with the wording:
\begin{verbatim}
  INSANE: bestCached != bestSoFar
\end{verbatim}
\Ticket{2245}

\item Fixed a bug introduced in HTCondor version 7.9.3,
in which concurrency limits were not respected across negotiation cycles when
\Macro{NEGOTIATOR\_CONSIDER\_PREEMPTION} was \Expr{False}.
\Ticket{3815}

\item Fixed a bug from HTCondor version 7.9.6. 
The bug exhibited itself when using CCB to connect to the \Condor{startd};
the \Condor{negotiator} and \Condor{schedd} would sometimes crash and then be restarted
with the following error message in the log:

\begin{verbatim}
ERROR "Selector::add_fd(): fd -1 outside valid range 0-1023"
\end{verbatim}

A workaround for the problem is relevant to HTCondor versions 7.9.6 through
8.0.1. Configure
\begin{verbatim}
  SERVICE_COMMAND_SOCKET_MAX_SOCKET_INDEX = -1
\end{verbatim}
\Ticket{3801}

\item Fixed a bug in the \Condor{qsub} script that caused it to exit
with a syntax error when a job with a memory requirement was
submitted.
\Ticket{3808}

\item Fix a bug causing security groups for EC2 jobs to be ignored.  
Also, the code respects the use of commas, as documented, 
to separate the items in the list of security groups specified by
the submit description file command \SubmitCmd{ec2\_security\_groups}. 
\Ticket{3787}

\item When invoking \Prog{glexec}, environment variable
\Env{GLEXEC\_TARGET\_PROXY} is now set to \File{/dev/null}.  
In some situations, it was previously set
to a nonexistent path, which caused errors in some configurations.
\Ticket{3794}

\item HTCondor daemons are now less vulnerable to long connection delays
when attempting to connect to hosts that are off-line.  A specific case
where this helps is when \Condor{schedd} is using a high availability
configuration, and the primary machine running the \Condor{collector} 
is off-line.
\Ticket{3828}

\item Fixed a bug that could cause \Condor{dagman} to hang 
due to a rarely seen event ordering.
This bug could have been triggered when using the
configuration variable \Macro{DAGMAN\_MAX\_JOBS\_IDLE},
or its equivalent command line option \Opt{-maxidle}.
\Ticket{3834}

\item Fixed a bug that caused job submission from Windows platforms
using \Condor{submit} with the \Opt{-spool} option to always fail.
\Ticket{3791}

\end{itemize}

\noindent Known Bugs:

\begin{itemize}

\item DAGMan recovery mode does not work for Pegasus-generated SUBDAGs.
For SUBDAGs, doing \Condor{hold} or \Condor{release} on
the \Condor{dagman} job, or stopping and re-starting the 
\Condor{schedd} with the DAGMan
job in the queue will result in failure of the DAG.  This can be
avoided by doing a \Condor{rm} of the DAGMan job, which produces a Rescue
DAG, and re-submitting the DAG; the Rescue DAG is automatically run.
This bug was introduced in HTCondor version 8.0.1.
\Ticket{3882}

\end{itemize}

\noindent Additions and Changes to the Manual:

\begin{itemize}

\item None.

\end{itemize}


%%%%%%%%%%%%%%%%%%%%%%%%%%%%%%%%%%%%%%%%%%%%%%%%%%%%%%%%%%%%%%%%%%%%%%
\subsection*{\label{sec:New-8-0-1}Version 8.0.1}
%%%%%%%%%%%%%%%%%%%%%%%%%%%%%%%%%%%%%%%%%%%%%%%%%%%%%%%%%%%%%%%%%%%%%%

\noindent Release Notes:

\begin{itemize}

\item HTCondor version 8.0.1 released on July 17, 2013.

\end{itemize}


\noindent New Features:

\begin{itemize}

\item HTCondor now provides the Debian Linux 7.0 (wheezy) platform,
including support for the standard universe.
\Ticket{3665}

\end{itemize}

\noindent Configuration Variable and ClassAd Attribute Additions and Changes:

\begin{itemize}

\item None.

\end{itemize}

\noindent Bugs Fixed:

\begin{itemize}

\item Fixed a bug that prevented per-slot settings of the 
\MacroNI{STARTD\_ATTRS} configuration variable from being set
correctly for partitionable slots named with a \Expr{SLOTX\_} prefix.
\Ticket{3726}

\item Fixed a bug that caused \Condor{status} \Opt{-submitters} to report twice
as many jobs running as were actually running. 
This bug appeared in HTCondor versions 7.9.6 and 8.0.0.
\Ticket{3713}

\item Fixed a bug with hierarchical group quotas in the \Condor{negotiator}
in which group hierarchies with parent groups that set 
configuration variable \Macro{GROUP\_ACCEPT\_SURPLUS} to
\Expr{False} would be assigned allocations above their quota.
\Ticket{3695}

\item Fixed a bug in which scheduler universe jobs that
have an \SubmitCmd{on\_exit\_hold}
expression that evaluates to \Expr{True} could have duplicate hold messages
in the user log.
\Ticket{3651}

\item Fixed a bug in which \Condor{dagman} would submit multiple copies of the
same job, fail, write a Rescue DAG, and leave the jobs in the queue. 
This was due to a warning from \Condor{submit} that the submit description file
was not using lines containing the string \Expr{"cluster"}. 
As a fix, \Condor{dagman} will search for the
string \Expr{" submitted to cluster "}.
This will generate fewer false alarms. 
If the submission succeeds and \Condor{dagman} gets confused, 
the jobs will be removed when \Condor{dagman} writes a Rescue DAG.
\Ticket{3658}

\item Added \Expr{libdate-manip-perl} as a dependency to the Debian packages.
It is required in order to run the \Condor{gather\_info} script.
\Ticket{3692}

\item Configuration variable \Macro{CCB\_ADDRESS} did not correctly 
support a list of CCB servers.  Only the first one in the list was used.
\Ticket{3699}

\item Fixed a bug that caused some communication layer log messages 
to end with binary characters.
\Ticket{3706}

\item Fixed a bug that can cause the \Condor{procd} to erroneously exit
on Mac OS X when many processes are created in a short period of time.
\Ticket{3725}

\item Removed a bug that caused \Condor{dagman} to have problems restarting
after an upgrade from HTCondor version 7.8.
\Ticket{3707}

\item Fixed a bug that caused the command 
\begin{verbatim}
  condor_q -dag -run
\end{verbatim}
to print garbage.
\Ticket{3578}

\item Fixed a bug that prevented jobs with an invalid \SubmitCmd{ec2\_keypair}
from being removed.
\Ticket{3485}

\item Fixed a memory leak and potential crash in the \Condor{gridmanager}
when requests to an EC2 service fail.
\Ticket{3701}

\item Fixed a bug in the \Condor{gridmanager} that can cause EC2 jobs to be
submitted a second time during recovery.
\Ticket{3705}

\item Fixed a memory leak in the \Condor{gridmanager} that was triggered when
submitting EC2 grid universe jobs.
\Ticket{3720}

\end{itemize}

\noindent Known Bugs:

\begin{itemize}

\item None.

\end{itemize}

\noindent Additions and Changes to the Manual:

\begin{itemize}

\item None.

\end{itemize}


%%%%%%%%%%%%%%%%%%%%%%%%%%%%%%%%%%%%%%%%%%%%%%%%%%%%%%%%%%%%%%%%%%%%%%
\subsection*{\label{sec:New-8-0-0}Version 8.0.0}
%%%%%%%%%%%%%%%%%%%%%%%%%%%%%%%%%%%%%%%%%%%%%%%%%%%%%%%%%%%%%%%%%%%%%%

\noindent Release Notes:

\begin{itemize}

%\item HTCondor version 8.0.0 not yet released.
\item HTCondor version 8.0.0 released on June 6, 2013.

\end{itemize}


\noindent New Features:

\begin{itemize}

% would have been in 7.9.7, but there was no 7.9.7 release
\item The \Condor{chirp} \Opt{write} command now accepts an 
optional \Arg{numbytes} parameter following the local file name.
This allows the write to be limited to the specified number of bytes.
\Ticket{3548}

% would have been in 7.9.7, but there was no 7.9.7 release
\item The HTCondor Python bindings now build on Mac OS X.
\Ticket{3584}

\item Updated the sample \File{condor.plist} file to work better with
current versions of Mac OS X.
\Ticket{3624}

\end{itemize}

\noindent Configuration Variable and ClassAd Attribute Additions and Changes:

\begin{itemize}

\item The new configuration variable
\Macro{DEDICATED\_SCHEDULER\_WAIT\_FOR\_SPOOLER}
permits the specification of a very strict execution order for 
parallel universe jobs handed to a remote scheduler.
\Ticket{2946}

\end{itemize}

\noindent Bugs Fixed:

\begin{itemize}

\item Fixed a bug that happened with partitionable slots, jobs that
requested more than one cpu, and a negotiator with
\Macro{NEGOTIATOR\_CONSIDER\_PREEMPTION} was false.  This would
cause the negotiator to incorrectly assume that each slot had
a slot weight of one.
\Ticket{3737}

\item The redundant configuration variable \Macro{CheckpointPlatform} has
been removed and the configuration variable \Macro{CHECKPOINT\_PLATFORM}
documented.
\Ticket{3544}

\item A standard universe job will no longer crash, and it will no longer 
cause the \Condor{shadow} to crash
if the job calls \Procedure{mmap} with an unsupported combination of flags.
\Ticket{3565}

\item Support for \Prog{VMware Workstation} and \Prog{VMware Player} 
under the \SubmitCmd{vm} universe now works properly on Windows platforms.
\Ticket{3627}

% would have been in 7.9.7, but there was no 7.9.7 release
\item For grid universe jobs intended for an EC2 grid resource,
errors which have no response body now report the HTTP code.
\Ticket{3541}

% would have been in 7.9.7, but there was no 7.9.7 release
\item \Condor{chirp} \Opt{put} would experience an assertion failure when
used on an empty file.  This bug has been fixed, and \Opt{put} can now be
used on an empty file.
\Ticket{3542}

% would have been in 7.9.7, but there was no 7.9.7 release
\item The 32-bit \Condor{starter} could fail to execute jobs when the initial
working directory of the job was on a subsystem containing 64-bit metadata,
such as inode numbers.
\Ticket{3605}

% would have been in 7.9.7, but there was no 7.9.7 release
\item \Condor{dagman} failed to react correctly if a nested DAG file
did not exist. It now reacts correctly and prints a more
helpful error message.
\Ticket{3623}

% would have been in 7.8.9, but there was no 7.8.9 release
\item Fixed a bug that caused the \Condor{master} daemon on Windows platforms
to think there were new binaries
when changing to and from daylight savings time.
The \Condor{master} daemon would then kill and restart itself,
as well as all of the daemons,
if configuration variable \Macro{MASTER\_NEW\_BINARY\_RESTART} was set
to its default value of \Expr{GRACEFUL}.
\Ticket{3572}

\item Fixed a bug that caused redundant lines to show up in the user log
at the end of the partitionable resource usage summary.
\Ticket{3621}

\item Fixed several bugs that can cause the \Condor{procd} to fail on Mac OS X
and not be restartable.
\Ticket{3617}
\Ticket{3618}
\Ticket{3620}

\item The \Condor{procd} now ignores process id 0 on Mac OS X.
\Ticket{3516}

\item Fixed memory leaks in the \Condor{shadow} and the \Condor{startd};
fixed a file descriptor leak in the standard universe \Condor{starter}.
\Ticket{3590}

\item Fixed a bug in which \Condor{dagman} would miscount the number
of held jobs when
multiple copies of hold events were written to the user log.
\Ticket{3650}

\end{itemize}

\noindent Known Bugs:

\begin{itemize}

\item The following obsolete binaries have not yet been removed from
the HTCondor tarballs:  
  \begin{itemize}
  \item \emph{classad\_functional\_tester}
  \item \emph{classad\_version}
  \item \Condor{test\_match}
  \item \Condor{userlog\_job\_counter}
  \end{itemize}
\Ticket{3670}

\item \Condor{status} \Opt{-submitters} reports twice
as many jobs running as were actually running.
\Ticket{3713}

\end{itemize}

\noindent Additions and Changes to the Manual:

\begin{itemize}

\item Fixed the \Condor{configure} man page and added a corresponding
\Condor{install} man page.
\Ticket{3619}

\item Added stub man pages for the Bosco commands.
\Ticket{3634}

\end{itemize}



%%%      PLEASE RUN A SPELL CHECKER BEFORE COMMITTING YOUR CHANGES!
%%%      PLEASE RUN A SPELL CHECKER BEFORE COMMITTING YOUR CHANGES!
%%%      PLEASE RUN A SPELL CHECKER BEFORE COMMITTING YOUR CHANGES!
%%%      PLEASE RUN A SPELL CHECKER BEFORE COMMITTING YOUR CHANGES!
%%%      PLEASE RUN A SPELL CHECKER BEFORE COMMITTING YOUR CHANGES!

%%%%%%%%%%%%%%%%%%%%%%%%%%%%%%%%%%%%%%%%%%%%%%%%%%%%%%%%%%%%%%%%%%%%%%
\section{\label{sec:History-7-9}Development Release Series 7.9}
%%%%%%%%%%%%%%%%%%%%%%%%%%%%%%%%%%%%%%%%%%%%%%%%%%%%%%%%%%%%%%%%%%%%%%

This is the development release series of HTCondor.
The details of each version are described below.

%% NO 7.9.7 RELEASE IS PLANNED.  PLEASE PUT YOUR ITEMS INTO THE 
%%  8.0.0 VERSION HISTORY   !!!

%%%%%%%%%%%%%%%%%%%%%%%%%%%%%%%%%%%%%%%%%%%%%%%%%%%%%%%%%%%%%%%%%%%%%%
\subsection*{\label{sec:New-7-9-6}Version 7.9.6}
%%%%%%%%%%%%%%%%%%%%%%%%%%%%%%%%%%%%%%%%%%%%%%%%%%%%%%%%%%%%%%%%%%%%%%

\noindent Release Notes:

\begin{itemize}

\item HTCondor version 7.9.6 released on May 8, 2013.

\end{itemize}


\noindent New Features:

\begin{itemize}

\item The new \Condor{ping} command line tool attempts one or more
targeted security negotiations to see if it succeeds or fails,
potentially helping to debug security configuration.
\Ticket{3371}

\item The \Condor{schedd} will now also advertise demand by jobs for slots,
weighted by the count of requested CPUs, if the configuration variable
\Macro{NEGOTIATOR\_USE\_WEIGHTED\_DEMAND} is set to \Expr{True}.
The default value is \Expr{False}.
\Ticket{3574}

\item Negotiation under groups now prefers the specification of
groups with the new submit commands \SubmitCmd{accounting\_group} and
\SubmitCmd{accounting\_group\_user}.
See section~\ref{sec:group-accounting} for details.
\Ticket{2728}

\item The \SubmitCmd{vm} universe now supports \Prog{VMware Workstation}
and \Prog{VMware Player}.
\Ticket{740}

\item On Linux platforms where cgroups are supported and enabled, the 
\Condor{starter} will now detect and trap if a vanilla universe job 
would otherwise be killed by the system Out Of Memory (OOM) killer.  
This situation is especially likely when a job sets \SubmitCmd{RequestMemory}
lower than needed.  The job will now be put on hold.
\Ticket{2992}

\item The new \Opt{-force-graceful} command-line option to \Condor{off}
allows administrators to issue a graceful shutdown command, even after
issuing a \Opt{-peaceful} command. Previously, a peaceful \Condor{off}
command would preclude a \Opt{-graceful} off command.
\Ticket{2949}

\item The \Condor{gather\_info} tool now includes the output of the Unix
\Prog{uptime} and \Prog{free} programs, 
as well as logs of the \Condor{master}, \Condor{startd}, and \Condor{starter}
of the machine where the job most recently ran, 
if \Condor{gather\_info} has the necessary permissions to fetch those logs.
\Ticket{3246}

\item The rotation of a daemon log file can now be specified 
in terms of time (seconds) or in terms of maximum size (bytes).
Only size was allowed previously.
\Ticket{3560}

\item The new \Condor{qsub} command line tool emulates submission to PBS, SGE, 
and Torque-like systems. It handles both scripts and command line options.
\Ticket{2699}

\item When submitting a grid universe job with a grid type of batch,
the value of \SubmitCmd{request\_memory} is now propagated to the batch system
submission request.
\Ticket{3398}

\item The set of Python bindings introduced in HTCondor version 7.9.4 is now
distributed as part of HTCondor, not as a contrib module.
\Ticket{3586}

\item Several improvements have been made to the
\Condor{gather\_info} tool.  It now prints the name of the
tarball it emits, and now also checks the history file for the
job in question, and if found, uses and displays the information 
there.
\Ticket{3239}
\Ticket{3240}

\end{itemize}

\noindent Configuration Variable and ClassAd Attribute Additions and Changes:

\begin{itemize}

\item The new configuration variable 
\Macro{GSI\_DELEGATION\_CLOCK\_SKEW\_ALLOWABLE}, expressed in seconds,
allows HTCondor to adjust the amount of
allowable clock skew between two parties.
This is relevant when delegating X.509 proxies.
\Ticket{3557}

\item The new configuration variable \Macro{CheckpointPlatform}
is a string that may be set by
an administrator to override the auto-detected
platform used to determine if a standard universe job that produced 
a checkpoint on one machine can be started on another.
\Ticket{3544}

%% Yes, SSSE 3 has three Ss.
\item The new Machine ClassAd attributes \AdAttr{Has\_sse4\_1},
\AdAttr{Has\_sse4\_2}, and \AdAttr{Has\_ssse3} are set to \Expr{True}
if the corresponding instruction set additions exist on that machine.
These attributes will be undefined otherwise.
\Ticket{3544}

\item The name of the configuration variable \MacroNI{MEMORY\_LIMIT}
introduced in HTCondor version 7.9.2 has changed.
This variable is now called \Macro{CGROUP\_MEMORY\_LIMIT\_POLICY}.
\Ticket{3564}

\item The new configuration variable \Macro{EXPIRE\_INVALIDATED\_ADS},
when set to \Expr{True}, causes invalidated ClassAds that would have
been removed from the \Condor{collector} right away to instead
be treated as expired ClassAds, such that they may become absent ClassAds.
See section~\ref{sec:Absent-Ads} for details on absent ClassAds.
\Ticket{3085}

\item The new configuration variable
\Macro{GLEXEC\_HOLD\_ON\_INITIAL\_FAILURE} controls whether jobs are put
on hold when a failure is encountered in the glexec setup phase of
managing the job.  The default value is \Expr{True}, 
which implements the previous behavior of putting a job on hold when
there is a failure.
\Ticket{3569}

\item The new configuration variable
\Macro{NEGOTIATOR\_CONSIDER\_EARLY\_PREEMPTION} controls whether jobs
can be matched to slots that still have retirement time remaining
before the existing job can be evicted.  The default is \Expr{False}.
The old behavior can be enabled by setting it to \Expr{True}.  The
new default behavior is intended to improve scheduling behavior
when \Expr{MaxJobRetirementTime} is used.
\Ticket{3539}

\item The new configuration variable \Macro{SCHEDD\_AUDIT\_LOG}
defines a file name, such that the
\Condor{schedd} can now write an audit log that records all
commands issued by users that modify the job queue.
\Ticket{3493}

\item The per-user file transfer I/O statistics now have a prefix of
\AdAttr{Owner\_<username>\_}.  In HTCondor version 7.9.5, 
they had a prefix of
\AdAttr{<username>\_}.  This can be configured via
\Macro{TRANSFER\_QUEUE\_USER\_EXPR}.
\Ticket{3496}

\item The new configuration variable
\Macro{BATCH\_GAHP\_CHECK\_STATUS\_ATTEMPTS} controls how often the
\Condor{gridmanager} should retry a failed job status check when using
the \Prog{batch\_gahp}. The default is 5.
\Ticket{3533}

\end{itemize}

\noindent Bugs Fixed:

\begin{itemize}

\item Fixed a bug in the setting of \Macro{ASSIGN\_CPU\_AFFINITY},
when running on systems with partitionable slots.
\Ticket{3597}

\item Fixed a bug in the \Condor{starter} in which file systems mounted by
the automounter would not be seen by the job.
\Ticket{3601}

\item Fixed the \Condor{updates\_stats} tool such that it works with
modern Perl interpreters.
\Ticket{3579}

\item Fixed a bug that caused the \Condor{schedd} to crash when 
\Condor{rm} was used with the \Opt{-f} option on parallel universe jobs.
\Ticket{3561}

\item Fixed a bug that could cause HTCondor-C jobs to fail when they are
part of a DAG. The jobs would be held with a hold reason of the form
\footnotesize
\begin{verbatim}
Failed to initialize user log to /dev/null or <DAG log path>.
\end{verbatim}
\normalsize
\Ticket{3474}

\item Fixed a severe leak of file descriptors in the \Condor{ft-gahp}.
\Ticket{3559}

\item Fixed a bug that occurred when using privilege separation;
the bug made it impossible for the \Condor{startd} to clean
up execute directories after a \Condor{starter} was prematurely killed.
\Ticket{3573}

\item Fixed a bug that sometimes caused fetching user priorities 
from the \Condor{negotiator} daemon to take a long time,
as the fetch potentially had to wait until the end of the negotiation cycle. 
The fetch no longer needs to wait.
\Ticket{3535}

\item Fixed a bug that would cause parallel universe jobs to fail when 
\Expr{USE\_NFS=True}. This might have caused a potential issue with 
doing a look up using \Condor{chirp}, 
although testing seems to show that this is not an issue.
\Ticket{3390}

\item Fixed a bug introduced in HTCondor version 7.9.4 that sometimes caused
attributes
\AdAttr{ExitCode} and \AdAttr{ExitBySignal} not to be set in the job
ClassAd when the job terminated.  These attributes were correctly
reported in the user log, but they were not propagated to the job ClassAd and
were therefore not available for querying with \Condor{history}.
\Ticket{3577}

\item Fixed a bug that caused the \Condor{starter} to leak file descriptors 
to file transfer plug-ins. 
\Ticket{3570}

\item Fixed a bug that caused the \Prog{batch\_gahp} to leak file descriptors 
when reading a job's X.509 proxy.
\Ticket{3558}

\item Fixed a bug with the dedicated scheduler reconnect mode.
If the \Condor{schedd} crashed with a parallel universe job running after the 
job's first job lease duration interval, parallel universe jobs would 
not successfully reconnect, and they
would restart from the beginning.
\Ticket{2291}

\end{itemize}

\noindent Known Bugs:

\begin{itemize}

\item Using \Condor{userprio} with the \Opt{-grouprollup} option
will fail to produce any output
if the \Condor{negotiator} it queries is of a version older 
than HTCondor version 7.9.6 and the \Condor{userprio} executable is 
HTCondor version 7.9.6.
\Ticket{3600}

\end{itemize}

\noindent Additions and Changes to the Manual:

\begin{itemize}

\item None.

\end{itemize}


%%%%%%%%%%%%%%%%%%%%%%%%%%%%%%%%%%%%%%%%%%%%%%%%%%%%%%%%%%%%%%%%%%%%%%
\subsection*{\label{sec:New-7-9-5}Version 7.9.5}
%%%%%%%%%%%%%%%%%%%%%%%%%%%%%%%%%%%%%%%%%%%%%%%%%%%%%%%%%%%%%%%%%%%%%%

\noindent Release Notes:

\begin{itemize}

\item HTCondor version 7.9.5 released on April 17, 2013.

\end{itemize}


\noindent New Features:

\begin{itemize}

\item The new command line tool \Condor{tail}
displays files that are in the sandbox of a running job.
See details in the manual page at
section~\ref{man-condor-tail}.
\Ticket{3522}

\item When there are multiple users
waiting to transfer files within the limits set by
configuration variables
\Macro{MAX\_CONCURRENT\_UPLOADS} and/or
\Macro{MAX\_CONCURRENT\_DOWNLOADS}, the scheduling algorithm
now gives the users
an equal share of the transfer slots.  How shares are counted can be
configured with \Macro{TRANSFER\_QUEUE\_USER\_EXPR}.
\Ticket{3487}

\item When using the \Opt{-remote} or \Opt{-spool} options to
\Condor{submit}, the job owner will now be set based
upon how the job submitter was authenticated. 
This will make it easier to submit jobs
to a remote \Condor{schedd} where the credentials may map to a different
account name.
\Ticket{3370} 

\item New functions are available in the Python Bindings contrib module.
ClassAds now more closely mimic Python dictionaries and provide
support for lists and values that are ClassAds. 
\Ticket{3494}

\item If a job is submitted specifying \SubmitCmd{keep\_claim\_idle},
the claim is kept not only when the job exits,
but also when the job is removed.
\Ticket{3491}

\item \Condor{dagman} now publishes in its own job ClassAd,
attributes with the DAG status,
such as total number of nodes, nodes queued, and nodes finished.
See section~\ref{sec:DAGStatusClassad} for more information.
\Ticket{1782}

\end{itemize}

\noindent Configuration Variable and ClassAd Attribute Additions and Changes:

\begin{itemize}

\item The new configuration variable \Macro{GSI\_DELEGATION\_KEYBITS}
allows the number of bits in a delegated proxy to be specified
by the receiving side.
\Ticket{3503}

\item When using file transfer concurrency limits, additional I/O
usage statistics are now published as attributes in the ClassAd of the
\Condor{schedd}.  This includes the sum and rate of bytes
transferred as well as time spent reading and writing to files and
to the network.  These statistics are reported for the sum of all
users and, when increased verbosity is configured, individually for
recently active users.  
These ClassAd attributes 

\begin{description}
  \item \AdAttr{FileTransferUploadBytes}
  \item \AdAttr{FileTransferUploadBytesPerSecond\_<timespan>}
  \item \AdAttr{FileTransferDownloadBytes}
  \item \AdAttr{FileTransferDownloadBytesPerSecond\_<timespan>}
  \item \AdAttr{FileTransferFileReadSeconds}
  \item \AdAttr{FileTransferFileReadLoad\_<timespan>} 
  \item \AdAttr{FileTransferFileWriteSeconds}
  \item \AdAttr{FileTransferFileWriteLoad\_<timespan>}
  \item \AdAttr{FileTransferNetReadSeconds}
  \item \AdAttr{FileTransferNetReadLoad\_<timespan>} 
  \item \AdAttr{FileTransferNetWriteSeconds}
  \item \AdAttr{FileTransferNetWriteLoad\_<timespan>} 
\end{description}
are fully described 
within the section on scheduler attributes at 
section~\pageref{sec:FT-Scheduler-ClassAd-Attributes}.
\Ticket{3496}

\item \Macro{NOT\_RESPONDING\_TIMEOUT} now internally adds some random skew
to avoid synchronization of heartbeat messages, which can lead to UDP
buffer overflow and incorrect determination that daemons are hung.
\Ticket{3510}

\end{itemize}

\noindent Bugs Fixed:

\begin{itemize}

\item The EC2 GAHP now treats OpenStack's \Code{stopped} state as if it
were \Code{shutoff}, terminating instances which enter this state and
preventing the instances from remaining in the queue forever.
\Ticket{3507}

\item Two EC2 GAHP bugs are fixed.
It now correctly parses XML namespaces as returned for
some installations of Eucalyptus.
The second bug caused HTCondor to put the job on hold, 
as it incorrectly believed that the cloud
service had purged it.
\Ticket{3492}

\item The EC2 GAHP now reports the
bidding status, defined at

\URL{http://docs.aws.amazon.com/AWSEC2/latest/UserGuide/using-spot-instances-bid-status.html},
for spot instances. 
\Ticket{3388}

\item The \Condor{negotiator} now checks to see if 
the time set by configuration variable 
\Macro{NEGOTIATOR\_MAX\_TIME\_PER\_SUBMITTER}
has been exceeded while negotiating with a single \Condor{schedd} daemon.
This configuration variable was previously only effective 
if a submitter used multiple \Condor{schedd} daemons.
\Ticket{3504}

\item \Condor{dagman} will now recover correctly in a DAG where a node has been 
skipped because of a \Macro{PRE\_SKIP} has triggered.
\Ticket{2966}

\item Fixed a bug in the \Condor{gridmanager} and \Condor{ft-gahp} that
could cause a crash when transferring files for grid universe jobs of 
grid type batch going to a remote cluster.
\Ticket{3529}

\item Fixed a bug in which \Condor{status} queries would not work,
with output of \AdStr{Access denied}, unless the \Condor{collector}
and the machine doing the query had synchronized clocks.
\Ticket{3360}

\item Fixed a Linux platform bug in which mount points were leaked
to the greater namespace when the configuration set
\MacroNI{MOUNT\_UNDER\_SCRATCH} for file systems
that have been mounted with shared propagation enabled.
\Ticket{3505}

\item Fixed a bug in the logging code that was causing grid universe batch jobs
to abort and drop a dprintf error file during file transfer once the log had
grown large enough to rotate.
\Ticket{3528}

\item On Windows platforms, running the \Condor{kbdd} no
longer creates a visible console window.
\Ticket{2805}

\end{itemize}

\noindent Known Bugs:

\begin{itemize}

\item Running \Condor{rm} with the \Opt{-f} option on a parallel universe
job can cause the \Condor{schedd} to crash.
\Ticket{3561}

\item Using privilege separation may cause execute directories to be leaked,
if the \Condor{starter} is shut down prematurely; 
for example, shut down may occur by a hard kill signal or power interruption.
\Ticket{3573}

\item  If a job has any output files and uses the file transfer mechanism,
the job ClassAd attribute \Attr{ExitCode} may be lost,
causing its value to be reported as 0.
\Ticket{3577}

\end{itemize}

\noindent Additions and Changes to the Manual:

\begin{itemize}

\item None.

\end{itemize}


%%%%%%%%%%%%%%%%%%%%%%%%%%%%%%%%%%%%%%%%%%%%%%%%%%%%%%%%%%%%%%%%%%%%%%
\subsection*{\label{sec:New-7-9-4}Version 7.9.4}
%%%%%%%%%%%%%%%%%%%%%%%%%%%%%%%%%%%%%%%%%%%%%%%%%%%%%%%%%%%%%%%%%%%%%%

\noindent Release Notes:

\begin{itemize}

\item HTCondor version 7.9.4 released on February 20, 2013.

\end{itemize}


\noindent New Features:

\begin{itemize}

\item Per job PID namespaces are available for Linux RHEL 6 platforms.
See section~\ref{sec:PIDNamespaces} for details.
\Ticket{1959}

\item The EC2 GAHP now batches requests for status updates, significantly
reducing its resource requirements.
\Ticket{3436}

\item The maximum total size of file transfers for a job may now be
specified using the new configuration variables
\Macro{MAX\_TRANSFER\_INPUT\_MB} and \Macro{MAX\_TRANSFER\_OUTPUT\_MB} 
and/or the new submit commands 
\SubmitCmd{max\_transfer\_input\_mb} and
\SubmitCmd{max\_transfer\_output\_mb}.
\Ticket{3333}

\item The \Prog{batch\_gahp} no longer relies on programs
\Prog{grid-proxy-info} and \Prog{grid-proxy-init} from the Globus
Toolkit to handle the X.509 proxies of jobs.
\Ticket{3431}

\item When the job's executable is transferred, always set the execute
bits on the copy.
\Ticket{3028}

\item By default, \Condor{dagman} now issues a fatal error
if any log file, which is either
the default log file or the log file specified for a node job,
is in \File{/tmp}, because this can cause DAGMan to fail.
This error can be downgraded to a warning by setting the
configuration variable
\MacroNI{DAGMAN\_USE\_STRICT} value to 0.
\Ticket{1419}

\item The \Condor{collector} will accept and display collector ClassAds for
multiple collectors from the same machine. For this to work, the
collectors must configured with different values for configuration
variable \MacroNI{COLLECTOR\_NAME}.
\Ticket{3467}

% Depending on who wins the argument between Igor and TJ, this might go into
% 7.8 series.  I think this is probably the right place for it, in 7.9. But
% Kent, TJ, or Igor may well have other ideas
\item \Condor{dagman} now will successfully set attributes for submitted jobs
using the \Condor{submit} syntax of placing a \Expr{+} sign just to the
left of the attribute name. 
See section~\ref{dagman:VARS} for more details.
\Ticket{3469}

\item The HTCondor contrib now includes a set of Python bindings in
two modules.
The \Code{htcondor} module interacts with the \Condor{schedd} and 
\Condor{collector} daemons. 
The \Code{classad} module provides an interface to work with ClassAds.
\Ticket{3407}

\item When using \Condor{compile}, 
Pthreads are not normally permitted to be used by standard universe jobs.
However, \Condor{compile} will now tell a user that they
should be linking to the GNU Pth library,
which is  built with the \verb@--enable-pthread@ flag.
This will permit jobs that use Pthreads to be built with \Condor{compile}.
\Ticket{3319}


\end{itemize}

\noindent Configuration Variable and ClassAd Attribute Additions and Changes:

\begin{itemize}

\item The new configuration variable \Macro{USE\_PID\_NAMESPACES}
enables per job PID namespaces for Linux RHEL 6 platforms when \Expr{True}.
\Ticket{1959}

\item The new configuration variable \Macro{FLOCK\_INCREMENT} allows
administrators to more aggressively flock to remote \Condor{collector} daemons,
as more pools will be considered.
\Ticket{3375}

\item The new configuration variable \Macro{HOST\_ALIAS} specifies the
  fully qualified host name that clients authenticating this daemon with 
  GSI should
  expect the daemon's certificate to match.  The alias is advertised
  to the \Condor{collector} as part of the address of the daemon.
  When this is not set, clients validate the daemon's certificate
  host name by matching it against DNS A records for the host they
  are connected to.  See \Macro{GSI\_SKIP\_HOST\_CHECK} for ways
  to disable this validation step.
\Ticket{1605}

\item The configuration variable \MacroNI{DAGMAN\_USE\_STRICT} now
defaults to a value of 1, rather than 0.
See the definition at section~\ref{param:DAGManUseStrict}.
\Ticket{3418}

\item The new configuration variable \Macro{GRACEFULLY\_REMOVE\_JOBS}
is a boolean value that controls whether jobs to be removed are 
gracefully removed.
The default is to do graceful removal.
\Ticket{3470}
\end{itemize}

\noindent Bugs Fixed:

\begin{itemize}

\item When HTCondor creates a key pair at an EC2 job's request, it no
longer fails to remove the private key from disk when the job leaves
the queue.
\Ticket{3477}

\item The EC2 GAHP now recognizes the OpenStack \Code{shutoff} state and
terminates instances which enter this state, 
preventing the instances from remaining in the queue forever.
\Ticket{3367}

\item \Condor{dagman} no longer does unnecessary sleeps for log file
consistency when a single default/workflow log file is used.
\Ticket{3456}

\item Fixed a bug introduced in HTCondor version 7.9.0 that caused 
the following configuration variables to not sort ClassAds properly 
when they evaluated to \Expr{True} or \Expr{False}:
\Macro{NEGOTIATOR\_PRE\_JOB\_RANK},
\Macro{NEGOTIATOR\_POST\_JOB\_RANK}, \Macro{PREEMPTION\_RANK}, and
\Macro{SCHEDD\_PREEMPTION\_RANK}.
\Ticket{3468}

\item Fixed a bug that can cause grid universe jobs of type \Expr{batch}
to fail when submitted to an HTCondor cluster with a large history file.
\Ticket{3429}

\item Corrected the submission of interactive jobs for cases 
in which the submit description file specified \SubmitCmd{Arguments}.
\Ticket{3455}

\item The semantics of signals sent to jobs were changed.
They have been changed back to the semantics defined in version 7.6.   
\Ticket{3470}

\end{itemize}

\noindent Known Bugs:

\begin{itemize}

\item None.

\end{itemize}

\noindent Additions and Changes to the Manual:

\begin{itemize}

\item None.

\end{itemize}

%%%%%%%%%%%%%%%%%%%%%%%%%%%%%%%%%%%%%%%%%%%%%%%%%%%%%%%%%%%%%%%%%%%%%%
\subsection*{\label{sec:New-7-9-3}Version 7.9.3}
%%%%%%%%%%%%%%%%%%%%%%%%%%%%%%%%%%%%%%%%%%%%%%%%%%%%%%%%%%%%%%%%%%%%%%

\noindent Release Notes:

\begin{itemize}

\item HTCondor version 7.9.3 released on January 16, 2013.

\end{itemize}


\noindent New Features:

\begin{itemize}

\item When the new configuration variable \Macro{ASSIGN\_CPU\_AFFINITY}
is set to \Expr{True}, 
the \Condor{startd} will automatically set the CPU affinity
mask jobs run with, so that a multi-threaded job will not use
more cores than the number it requests.
\Ticket{3348}

\item When configuration variable \Macro{NEGOTIATOR\_CONSIDER\_PREEMPTION}
is \Expr{False}, the \Condor{negotiator}
now fetches machine ClassAds more quickly from the \Condor{collector}
 by skipping most attributes of the busy machines.  
This can make negotiation much faster in
a very large pool of mostly claimed machines.
\Ticket{3366}

\item Round-robin scheduling is now used when there are multiple users
waiting to transfer files in the limits set by
\Macro{MAX\_CONCURRENT\_UPLOADS} and/or
\Macro{MAX\_CONCURRENT\_DOWNLOADS}.  Previously, the file transfer
queue was scheduled in first-in-first-out order, so one user with
many files to transfer could delay other users for as long as it took
to transfer those files.  Now, when choosing a new job to allow to
transfer, the first job belonging to the user who has least
recently been given an opportunity to transfer will be selected.
The old behavior, or variations on the new behavior, can be achieved
by configuring \Macro{TRANSFER\_QUEUE\_USER\_EXPR}.
\Ticket{3333}

\item \Condor{dagman} will now try twice to write a POST script terminate
event, rather than trying once and exiting. 
If it is unable to write the event, \Condor{dagman} exits, 
writing a Rescue DAG. 
\Ticket{965}

\item The \Condor{gridmanager} now cleans up temporary files and directories
that are sometimes left by the \Prog{batch\_gahp} when executing a grid
universe job of grid type \SubmitCmd{batch}.
\Ticket{3276}

\item Added counts of nodes in various states to the \Condor{dagman}
node status file.  Refer to section~\ref{sec:DAG-node-status} for
more information.
\Ticket{2075}

\item When submitting jobs to a remote batch system (for example, Bosco),
file transfer no longer requires a network connection from the remote machine
back to the local one.
\Ticket{3293}

\end{itemize}

\noindent Configuration Variable and ClassAd Attribute Additions and Changes:

\begin{itemize}

\item The new expert-only configuration variable 
\Macro{STATISTICS\_WINDOW\_QUANTUM}
allows administrators to set the time interval, 
known as a quantum, that divides a window over which statistics are
kept into smaller pieces.  The window advances one quantum at a time. 
\Ticket{3288}


\end{itemize}

\noindent Bugs Fixed:

\begin{itemize}

\item Jobs of the EC2 grid type which make invalid requests of the
service no longer go on hold when removed.
An example of this is when a job specifies a nonexistent AMI. 
\Ticket{3287}

\item Jobs of the EC2 grid type which cannot authenticate with the
service no longer go on hold when removed.
\Ticket{3387}

\item Fixed a problem with \Prog{glexec} that caused jobs not to start 
due to permission errors on the execute directory.
\Ticket{3369}

\item A change was made to more accurately implement the
minimum time defined by the configuration variable
\Macro{NEGOTIATOR\_CYCLE\_DELAY}. 
\Ticket{3332}

\item The \Prog{batch\_gahp} is no longer dependent on the Perl module 
\Code{XML::Simple} when submitting jobs to SGE.
\Ticket{3350}

\item The \Prog{batch\_gahp} now properly handles job X.509 proxies that 
are not in the old proxy format.
\Ticket{3362}

\item On 32-bit platforms,
setting configuration variable \Macro{STARTER\_RLIMIT\_AS} to a value 
larger than 4096 could cause jobs to abort on start up.
Since values larger than 2047 have no real meaning on 32-bit platforms,
the fix treats values larger than 2047 as no limit on 32-bit platforms.
\Ticket{3309}

\item Fixed a bug that can cause proxy refresh to fail for pbs, lsf,
and sge grid jobs.
\Ticket{3383}

\item When doing remote pbs, lsf, or sge grid job submissions, the
\Condor{gridmanager} now ensures that no unusual characters are used in
the name of the job sandbox directory it creates.
\Ticket{3294}

\item When a GAHP server fails to start, the \Condor{gridmanager} now
puts the affected jobs on hold.
\Ticket{3301}

\item Environment variable \Env{GLOBUS\_LOCATION} is now set for
\Prog{batch\_gahp},
allowing it to find proxy management that it needs for jobs that have an
X.509 proxy.
\Ticket{3015}

\item The installation RPM now requires Security Enhanced Linux (SELinux) 
scripts at post install time,
so that the scripts can set the appropriate security contexts.
\Ticket{3313}

\end{itemize}

\noindent Known Bugs:

\begin{itemize}

\item None.

\end{itemize}

\noindent Additions and Changes to the Manual:

\begin{itemize}

\item Initial documentation for EC2 spot instances can be found
in section~\ref{sec:spot-instances}.
\Ticket{3209}

\end{itemize}


%%%%%%%%%%%%%%%%%%%%%%%%%%%%%%%%%%%%%%%%%%%%%%%%%%%%%%%%%%%%%%%%%%%%%%
\subsection*{\label{sec:New-7-9-2}Version 7.9.2}
%%%%%%%%%%%%%%%%%%%%%%%%%%%%%%%%%%%%%%%%%%%%%%%%%%%%%%%%%%%%%%%%%%%%%%

\noindent Release Notes:

\begin{itemize}

%\item HTCondor version 7.9.2 not yet released.
\item HTCondor version 7.9.2 released on December 11, 2012.
This release contains all of the bug fixes in the version 7.8.6 
stable release,
and most of the bug fixes in the
soon to be released version 7.8.7 stable release.

\end{itemize}


\noindent New Features:

\begin{itemize}

\item The permissions for the temporary execute directory of a job
have been tightened for vanilla universe jobs, 
such that only the owner of the job is allowed to see or
modify the contents.
\Ticket{3315}

\item Added experimental support for EC2 spot instances.
\Ticket{3209}

\item (This feature was added in version 7.9.1.)  
There are two new protocols for the submission of grid type EC2 jobs,
\Expr{euca3://} and \Expr{euca3s://}.
These protocols exist to work correctly when the resources do not support 
the \Param{InstanceInitiatedShutdownBehavior} parameter.
\Ticket{2974}

\item (This feature was added in version 7.9.1.)  
Added both a \Opt{-suppress\_notification},
a \Opt{-dont\_suppress\_notification} command line option,
and corresponding
\Macro{DAGMAN\_SUPPRESS\_NOTIFICATION} configuration variable
to \Condor{dagman} and \Condor{submit\_dag}.
This enables a user of DAGMan to stop email notification of job
events for jobs submitted by \Condor{dagman}. The value of
\MacroNI{DAGMAN\_SUPPRESS\_NOTIFICATION} defaults to \Expr{True},
so that jobs submitted
by \Condor{dagman} will not send email notification. 
\Ticket{3352}

\item The default for job notification email has changed
from \Expr{Complete} to \Expr{Never}. 
There is also a new configuration variable, \Macro{JOB\_DEFAULT\_NOTIFICATION},
which permits administrators to change the default for all jobs.
\Ticket{2155}

\item For platforms supporting cgroups,
resource limits can now be applied per job,
where a job may consist of multiple processes.
See section~\ref{sec:Resource-Limits-Cgroup} for details.
\Ticket{2734}

\end{itemize}

\noindent Configuration Variable and ClassAd Attribute Additions and Changes:

\begin{itemize}

\item The new configuration variable \Macro{MEMORY\_LIMIT}
supports implementing memory resource limits on a per-job basis under cgroups.
\Ticket{2734}

\end{itemize}

\noindent Bugs Fixed:

\begin{itemize}

\item \Condor{schedd} and \Condor{shadow} were not respecting the
\Macro{DAGManNodesMask} attribute. This caused extra events to be written to
the DAGMan node log.
\Ticket{3311}

\item Removed a spurious newline from the output of \Condor{submit}.
\Ticket{3316}

\item Fixed a bug that caused the \Condor{shadow} to set job attribute
\Attr{X509UserProxySubject} to the wrong value when the job's X.509
proxy file was updated. It incorrectly set the value to be 
the proxy's subject name, rather than to the correct value, which is
its identity.
\Ticket{3265}

\item The \Prog{batch\_gahp} no longer modifies the environment variable
\Env{LD\_LIBRARY\_PATH}.
In some instances, modifying \Env{LD\_LIBRARY\_PATH} caused the
batch system's command line tools to fail when run by the \Prog{batch\_gahp}.
\Ticket{3317}

\item Grid-type \SubmitCmd{batch} jobs now work properly on machines
where the gLite software has been installed.
\Ticket{3269}

\item The \Condor{shadow} would never print the allocated amount of
partitionable resources in the job log.
\Ticket{3318}

\item \Condor{who} would sometimes incorrectly display blank or partial
values in the PROGRAM column.
\Ticket{3314}

\end{itemize}

\noindent Known Bugs:

\begin{itemize}

\item None.

\end{itemize}

\noindent Additions and Changes to the Manual:

\begin{itemize}

\item None.

\end{itemize}


%%%%%%%%%%%%%%%%%%%%%%%%%%%%%%%%%%%%%%%%%%%%%%%%%%%%%%%%%%%%%%%%%%%%%%
\subsection*{\label{sec:New-7-9-1}Version 7.9.1}
%%%%%%%%%%%%%%%%%%%%%%%%%%%%%%%%%%%%%%%%%%%%%%%%%%%%%%%%%%%%%%%%%%%%%%

\noindent Release Notes:

\begin{itemize}

\item Condor version 7.9.1 released on October 22, 2012.

\item Condor no longer looks for its main configuration file in the
location \File{\MacroUNI{GLOBUS\_LOCATION}/etc/condor\_config}.
\Ticket{2830}

\item \Security This version contains an important security bug fix.  See below
for details of this and other bugs fixed.

\end{itemize}


\noindent New Features:

\begin{itemize}

\item There are two new protocols for the submission of grid type EC2 jobs,
\Expr{euca3://} and \Expr{euca3s://}.
These protocols exist to work correctly when the resources do not support 
the \Param{InstanceInitiatedShutdownBehavior} parameter.
\Ticket{2974}

\item \Condor{job\_router} can now submit the routed copy of jobs to a
different \Condor{schedd} than the one that serves as the source of
jobs to be routed.  The spool directories of the two
\Condor{schedds} must still be directly accessible to
\Condor{job\_router}.  This feature is enabled by using the new
optional configuration settings:

\begin{itemize}
\item \Macro{JOB\_ROUTER\_SCHEDD1\_SPOOL}
See definition at section~\ref{param:JobRouterSchedd1Spool}.
\item \Macro{JOB\_ROUTER\_SCHEDD2\_SPOOL}
See definition at section~\ref{param:JobRouterSchedd2Spool}.
\item \Macro{JOB\_ROUTER\_SCHEDD1\_NAME}
See definition at section~\ref{param:JobRouterSchedd1Name}.
\item \Macro{JOB\_ROUTER\_SCHEDD2\_NAME}
See definition at section~\ref{param:JobRouterSchedd2Name}.
\item \Macro{JOB\_ROUTER\_SCHEDD1\_POOL}
See definition at section~\ref{param:JobRouterSchedd1Pool}.
\item \Macro{JOB\_ROUTER\_SCHEDD2\_POOL}
See definition at section~\ref{param:JobRouterSchedd2Pool}.
\end{itemize}
\Ticket{3030}

\item The \Condor{job\_router} can now optionally transform jobs in place,
rather than creating a second transformed version (copy) of the job.
\Ticket{3185}

\item The \Condor{defrag} daemon now has a policy option implemented
by configuration to cancel the draining
of a machine that is in the Draining mode.  This can be used to effect
partial draining of machines.
\Ticket{2993}

\item Communication between the \Condor{c-gahp} and the \Condor{schedd} has
been improved. A large number of Condor-C jobs should no longer cause
other clients of the remote \Condor{schedd} to time out trying to get the
\Condor{schedd} daemon's attention.
\Ticket{2575}

\item \Condor{history} and \Condor{q} can now be told to read job records
from a user log, instead of parsing the history file or querying the
\Condor{schedd}.  This can be used to monitor the status of jobs with
reduced load on the \Condor{schedd}.
\Ticket{3188}

\item Eucalyptus 3.x support has been added to the EC2 GAHP.
\Ticket{2974}

\item File transfer remaps now support remapping directories.
\Ticket{3039}

\item The \Condor{schedd} can now dynamically spawn a local \Condor{startd}
to manage local universe jobs.
\Ticket{3129}

\item \Condor{q} \Opt{-jobads} will now respect the \Opt{-constraint} option.
\Ticket{3191}

\item Added Bosco, a set of tools that makes it easy to use a Personal
Condor to run jobs on remote batch systems without administrator
assistance or manual installation of software on the remote systems.
See \URL{https://twiki.grid.iu.edu/bin/view/CampusGrids/BoSCO} for more
information about Bosco.
\Ticket{2421}

\end{itemize}


\noindent Configuration Variable and ClassAd Attribute Additions and Changes:

\begin{itemize}

\item Dynamic slots now fill the values for attributes of with names
that begin with
\Attr{TotalSlot}, 
for configured local resources in a way consistent with standard resources
such as \Attr{TotalSlotCpus}.
Previously those values were all given the value zero on dynamic slots.
\Ticket{3229}

\item The \Condor{schedd} now advertises the value of configuration variable
\MacroNI{COLLECTOR\_HOST} as attribute \Attr{CollectorHost} in 
its daemon ClassAd.  This allows one to determine if a given
\Condor{schedd} reporting to a \Condor{collector} is flocking to that 
\Condor{collector} or not.
\Ticket{3202}

\item Added the attribute \Attr{DAGManNodesMask} to control the verboseness of
the log referred to by \Attr{DAGManNodesLog}.
\Ticket{3351}

\item The new configuration variable
\Macro{QUEUE\_SUPER\_USER\_MAY\_IMPERSONATE} specifies a regular
expression that matches the user names that
the queue super user may impersonate when managing jobs.  When not
set, the default behavior is to allow impersonation of any user who
has had a job in the queue during the life of the \Condor{schedd}.  For
proper functioning of the \Condor{shadow}, the \Condor{gridmanager}, and
the \Condor{job\_router}, this expression, if set, must match the owner
names of all jobs that these daemons will manage.
\Ticket{3030}

\item The new configuration variable \Macro{DEFRAG\_CANCEL\_REQUIREMENTS}
is an expression that specifies which draining machines should have 
draining be canceled.  
This defaults to \MacroUNI{DEFRAG\_WHOLE\_MACHINE\_EXPR}.  
This could be used to drain partial rather than whole machines.
\Ticket{2993}

\item The new submit command \SubmitCmd{use\_x509userproxy} can be set
to \Expr{True} to indicate that an X.509 user proxy is required for the job. 
If \SubmitCmd{x509userproxy} is not set, 
then the proxy file will be looked for in the standard locations.
\Ticket{3025}

\item If \Condor{submit} is used to submit an interactive job,
and the job is interrupted before the interactive job starts,
an attempt is made to immediately remove the interactive job from the queue.
Similarly, \Condor{ssh\_to\_job} has a new option \Opt{-remove-on-interrupt}.
\Ticket{3242}

\item Changes to were made to the ClassAd machine attributes 
\Attr{OpSys}, \Attr{OpSysVer}, \Attr{Distro}, as well as others,
in order to do a better job of identifying the operating system.
\Ticket{2366}

\item \Macro{GRIDMANAGER\_MAX\_SUBMITTED\_JOBS\_PER\_RESOURCE} can now be a
list, specifying different values for different hosts.
\Ticket{3220}

\item The new configuration parameter \Macro{GRIDMANAGER\_JOB\_PROBE\_RATE}
limits the number of job status requests sent to each remote resource.
\Ticket{3023}

\item The default value of \Macro{GRIDMANAGER\_JOB\_PROBE\_INTERVAL} has
changed from 300 to 60.
\Ticket{3023}.

\item The configuration parameters \Macro{CONDOR\_JOB\_POLL\_INTERVAL} and
\Macro{INFN\_JOB\_POLL\_INTERVAL} should no longer be used. Use
\Macro{GRIDMANAGER\_JOB\_PROBE\_INTERVAL\_CONDOR} and
\Macro{GRIDMANAGER\_JOB\_PROBE\_INTERVAL\_BATCH} instead.
\Ticket{3023}

\end{itemize}

\noindent Bugs Fixed:

\begin{itemize}

\item \Security Fixed a bug which allowed jobs submitted to the standard
universe to escalate privilege on the submit machine and execute code as 
\Login{root}.
(CVE-2012-5390)
\Ticket{3268}

\item A fix only invokes Globus callouts when actually needed, 
thereby avoiding a program segfault if
the call out mechanism is misconfigured or broken.
\Ticket{2104}

\item Fixed a bug in all daemons wherein the \Attr{DaemonStartTime} attribute 
in the ClassAd for all daemons would be reset to the current time when they are
reconfigured.
\Ticket{3235}

\item Fixed a bug wherein the \Arg{-dont\_use\_default\_node\_log} command line flag
to \Condor{submit\_dag} had no effect.
\Ticket{3352}

\item \Security Although not user-visible, 
there were multiple updates that removed places
in the code where potential buffer overruns could occur, 
thus preventing potential attacks.  
None of these overruns were known to be exploitable.

\item \Security Although not user-visible, 
there were updates to the code to improve
the error checking of system calls,
thereby removing some potential security threats.  
None were known to be exploitable.

\item \Security Although not user-visible, 
removed some code that was no longer used.
The presence of this code could have led to a Denial-of-Service attack,
which would allow an attacker to stop another user's jobs from running.

\item \Security Filesystem (FS) authentication was improved to check 
the UNIX permissions of the directory used for authentication.  
Without this, an attacker may have
been able to impersonate another submitter on the same submit machine.

\item The \Condor{negotiator} now checks the accountant log file for sanity
once only on start up,  
thereby increasing efficiency of iteration through 
the accountant ClassAd log structure.
\Ticket{3011}

\item The ClassAd functions \Procedure{splitUserName} and 
\Procedure{splitSlotName}
no longer leak a small amount of memory each time they are evaluated.  
This bug was introduced when these functions were added in Condor version 7.7.6.
\Ticket{3082}

\item There are several bug fixes for grid-type batch jobs:
  \begin{itemize}
  \item Monitoring the status of jobs submitted to PBS and SGE has been
    improved. \Ticket{3067} \Ticket{3157} \Ticket{3181}
  \item Job command-line arguments containing 
    left parenthesis, \verb@(@, right parenthesis, \verb@)@, 
    and ampersand, \verb@&@, characters are now handled properly. 
    \Ticket{3057}
  \item Removing PBS jobs that have just completed no longer causes the jobs
    to become held. \Ticket{3016}
  \item Added a work-around for a bug when submitting jobs to
    a Condor pool running Condor versions 7.7.6 through 7.8.2.
    A bug in \Condor{history} \Opt{-f} caused an error in determining
    a job's status.
    \Ticket{3133}
  \item Improved the handling of job files when the batch system has a shared
    file system. \Ticket{3195}
  \end{itemize}

\item Changes introduced in Condor version 7.9.0 caused jobs submitted by
\Condor{dagman} in the local universe to not write to the default node log file,
when \Macro{DAGMAN\_ALWAYS\_USE\_NODE\_LOG} was \Expr{True} (the default),
and a user log was also defined. This is fixed. 
\Ticket{3111}

\item Fixed a bug introduced in Condor version 7.9.0 that caused grid type
cream jobs to be held with a hold reason of 
\footnotesize
\begin{verbatim}
  CREAM_Delegate Error: Cannot set credentials in the gsoap-plugin context.
\end{verbatim}
\normalsize
\Ticket{3234}

\item Fixed a problem that could have caused the \Condor{collector} to crash
when receiving an invalid packet.
\Ticket{3161}

\end{itemize}

\noindent Known Bugs:

\begin{itemize}

\item None.

\end{itemize}

\noindent Additions and Changes to the Manual:

\begin{itemize}

\item None.

\end{itemize}


%%%%%%%%%%%%%%%%%%%%%%%%%%%%%%%%%%%%%%%%%%%%%%%%%%%%%%%%%%%%%%%%%%%%%%
\subsection*{\label{sec:New-7-9-0}Version 7.9.0}
%%%%%%%%%%%%%%%%%%%%%%%%%%%%%%%%%%%%%%%%%%%%%%%%%%%%%%%%%%%%%%%%%%%%%%

\noindent Release Notes:

\begin{itemize}

\item Condor version 7.9.0 released on August 16, 2012.

\end{itemize}


\noindent New Features:

\begin{itemize}

\item Machine slots can now be configured to identify and
divide customized local resources.
Jobs may then request these resources.
See section~\ref{sec:Configuring-SMP} for details.
\Ticket{2905}

\item Condor now supports and implements the caching of ClassAds 
to reduce memory footprints. 
This feature is experimental and is currently disabled by default.
It can be enabled by setting
the new configuration variable \Macro{ENABLE\_CLASSAD\_CACHING}
to \Expr{True}.
\Ticket{2541}
\Ticket{3127}

\item \Condor{status} now returns the \Condor{schedd} ClassAd directly 
from the \Condor{schedd} daemon,
if both options \Opt{-direct} and \Opt{-schedd} are given on the command line.
\Ticket{2492}

\item The new \Opt{-status} and \Opt{-echo} command line options to 
\Condor{wait} command cause it to show job start and terminate information,
and to print events to \Code{stdout}.
\Ticket{2926}

\item Added a \Expr{DEBUG} logging level output flag \Dflag{CATEGORY},
which causes Condor to include the logging level
flags in effect for each line of logged output.
\Ticket{2712}

\item \Condor{status} and \Condor{q} each have a new \Opt{-autoformat} option
to make some output format specifications easier than the existing
\Opt{-format} option.
See the \Condor{status} manual page located on page~\pageref{man-condor-status}
and the \Condor{q} manual page located on page~\pageref{man-condor-q} 
for details.
\Ticket{2941}

\item Enhanced the ClassAd log system to report the log line number 
on parse failures, 
and improved the ability to detect parse failures closer to 
the point of corruption.
\Ticket{2934}

\item Added an \Opt{-evaluate} option to \Condor{config\_val}, which causes the configured value queried from
a given daemon to be evaluated with respect to that daemon's ClassAd.
\Ticket{856}

\item Added code to \Condor{dagman},
such that a \Expr{VARS} assignment in a top-level DAG is applied to splices.
\Ticket{1780}

\item Condor now uses libraries from Globus 5.2.1.
\Ticket{2838}

\item When authenticating Condor daemons with GSI and
configuration variable \MacroNI{GSI\_DAEMON\_NAME} is undefined, 
Condor checks that the server name in the certificate matches the 
host name that the client is connecting to. 
When \MacroNI{GSI\_DAEMON\_NAME} is defined,
the old behavior is preserved: only certificates matching
\MacroNI{GSI\_DAEMON\_NAME} pass the authentication step, 
and no host name check is performed.  
The behavior of the host name check
may be further controlled with the new configuration variables
\MacroNI{GSI\_SKIP\_HOST\_CHECK} and
\MacroNI{GSI\_SKIP\_HOST\_CHECK\_CERT\_REGEX}.
\Ticket{1605}

\item Added new capability to \Condor{submit} to allow recursive macros in
submit description files. 
This facility allows one to update variables recursively. 
Before this new capability was added,
recursive definition would send \Condor{submit} into an
infinite loop of expanding the macro,
such that the expansion would fill up memory.
See section~\ref{macro-in-submit-description-file} for details.
\Ticket{406}

\item A DAGMan limitation and restriction has been removed.  
It is now permitted to define a \SubmitCmd{log} command using a macro,
within a node job's submit description file.
\Ticket{2428}

\item To enforce the dependencies of a DAG,
DAGMan now uses and watches only the default node
user log of the \Condor{dagman} job for events.  
DAGMan requests the \Condor{schedd} and \Condor{shadow} daemons to write each
event to this default log, 
in addition to writing to a log specified by the node job.
\Condor{dagman} writes POST script terminate events only to its default log;
these terminate events are not written to the user log.
This behavior can be turned off by setting the configuration variable
\Macro{DAGMAN\_ALWAYS\_USE\_NODE\_LOG} to \Expr{False}.
For correct behavior,
\MacroNI{DAGMAN\_ALWAYS\_USE\_NODE\_LOG} should be set to \Expr{False}
if \Condor{dagman} version 7.9.0 or later is submitting jobs 
to an older version of
a \Condor{schedd} daemon or of a \Condor{submit} executable.
\Ticket{2807}

\item \Condor{submit} has a new \Opt{-interactive} option for
platforms other than Windows,
which schedules and runs a job that provides a shell prompt
on the execute machine.
\Ticket{3088}

\end{itemize}

\noindent Configuration Variable and ClassAd Attribute Additions and Changes:

\begin{itemize}

\item The new configuration variables \Macro{MACHINE\_RESOURCE\_NAMES}
(see section~\ref{param:MachineResourceNames})
and \Macro{MACHINE\_RESOURCE\_<name>}
(see section~\ref{param:MachineResourceResourcename})
identify and specify the use of customized local machine resources.
\Ticket{2905}

\item The new configuration variable \MacroNI{ENABLE\_CLASSAD\_CACHING}
controls whether the new caching feature of ClassAds is used.
The default value is \Expr{False}.
\Ticket{3127}

\item The new configuration variable \Macro{CLASSAD\_LOG\_STRICT\_PARSING}
controls whether ClassAd log files such as the job queue
log are read with strict parse checking on ClassAd expressions.
\Ticket{3069}

\item The default value for configuration variable \Macro{USE\_PROCD}
is now \Expr{True} for the \Condor{master} daemon.  
This means that by
default the \Condor{master} will start a \Condor{procd} daemon to be used 
by it and all other daemons on that machine.
\Ticket{2911}

\item There is a new configuration variable used by the \Condor{starter}.
If \Macro{STARTER\_RLIMIT\_AS} is set to an integer value, 
the \Condor{starter}
will use the \Procedure{setrlimit} system call with the 
\Code{RLIMIT\_AS} parameter to
limit the virtual memory size of each process in the user job.  
The value of this configuration variable is a ClassAd expression, 
evaluated in the context of both the machine and job ClassAds, 
where the machine ClassAd is the \Expr{my} ClassAd, 
and the job ClassAd is the \Expr{target} ClassAd.
\Ticket{1663}

\item New configuration variables were added to to the \Condor{schedd} to
define statistics that count subsets of jobs. 
These variables have the form \Macro{SCHEDD\_COLLECT\_STATS\_BY\_<Name>},
and should be defined by a ClassAd expression that evaluates to a string.
See section~\ref{param:ScheddCollectStatsByName}
for the complete definition.
The optional configuration variable of the form
\Macro{SCHEDD\_EXPIRE\_STATS\_BY\_<Name>} can be used to set an expiration time,
in seconds, for each set of statistics.
\Ticket{2862}

\item The new \SubmitCmd{batch\_queue} submit description file command
and new job ClassAd attribute \Attr{BatchQueue} specify which job
queue to use for grid universe jobs of type
\SubmitCmd{pbs}, \SubmitCmd{lsf}, and \SubmitCmd{sge}.
\Ticket{2996}

\item The new configuration variable \Macro{GSI\_SKIP\_HOST\_CHECK} is
a boolean that controls whether a check is performed during
GSI authentication of a Condor daemon.  
When the default value \Expr{False},
the check is not skipped, so the daemon host name must match the
host name in the daemon's certificate, unless otherwise exempted
by values of \MacroNI{GSI\_DAEMON\_NAME} or
\MacroNI{GSI\_SKIP\_HOST\_CHECK\_CERT\_REGEX}.
When \Expr{True}, this check is skipped, and hosts will not be rejected
due to a mismatch of certificate and host name.
\Ticket{1605}

\item The new configuration variable
\MacroNI{GSI\_SKIP\_HOST\_CHECK\_CERT\_REGEX} may be set to a
regular expression.  GSI certificates of Condor daemons with a
subject name that are matched in full by this regular expression
are not required to have a matching daemon host name and certificate
host name.  The default is an empty regular expression, which will
not match any certificates, even if they have an empty subject name.
\Ticket{1605}

\end{itemize}

\noindent Bugs Fixed:

\begin{itemize}

\item Fixed a bug in which usage of cgroups incorrectly included the page cache 
in the maximum memory usage.
This bug fix is also included in Condor version 7.8.2.
\Ticket{3003}

\item The EC2 GAHP will now respect the value of the environment variable
\Env{X509\_CERT\_DIR} and the configuration variable
\Macro{GSI\_DAEMON\_TRUSTED\_CA\_DIR} for \emph{all} secure connections.
\Ticket{2823}

\item Condor will avoid selecting down (disabled) network interfaces.  Previously Condor could select a down interface over an up (active) interface.
\Ticket{2893}

\item Made logic in the \Condor{negotiator} that computes submitter limits 
properly aware of the configuration variable
\Macro{NEGOTIATOR\_CONSIDER\_PREEMPTION}.
\Ticket{2952}


\item Condor no longer back-dates file modification times by 3 minutes
when transferring job input files into the job spool directory or the job
execute directory.
\Ticket{2423}

\item Fixed a bug in which the use of a pipe in the configuration file 
on Windows platforms would cause a visible console window 
to show up whenever the configuration was read.
\Ticket{1534}

\end{itemize}

\noindent Known Bugs:

\begin{itemize}

\item None.

\end{itemize}

\noindent Additions and Changes to the Manual:

\begin{itemize}

\item Machine ClassAd attribute string values relating to \Attr{OpSys} have
been documented for Scientific Linux platforms.
\Ticket{2366}

\end{itemize}


\input{version-history/7-8.history.tex}
% as of the 8.0.0 release, the 7-7 and 7-6 version histories no longer included.
%\input{version-history/gotchas.tex}
%\input{version-history/7-7.history.tex}
%\input{version-history/7-6.history.tex}
% as of April 2012, Karen no longer wants to include these older
% version histories with the 7.4 and 7.5 manuals.
%\input{version-history/7-5.history.tex}
%\input{version-history/7-4.history.tex}
% as of April 2011, Karen no longer wants to include these older
% version histories with the 7.6 and beyond manuals.
%\input{version-history/7-3.history.tex}
%\input{version-history/7-2.history.tex}
%\input{version-history/7-1.history.tex}
%\input{version-history/7-0.history.tex}
% Oct 2009, as we release 7.4, Karen commented out inclusion of the
% 6.9 and 6.8 histories
%\input{version-history/6-9.history.tex}
%\input{version-history/6-8.history.tex}
% Dec 2007, as we release 7.x, Karen commented out the older histories
%\input{version-history/6-7.history.tex}
%\input{version-history/6-6.history.tex}
% Feb 2007 -- still in the manual source, just not incorporating
% these old histories into the finished product, thereby
% reducing the size of the manual by 200 pages
%\input{version-history/6-5.history.tex}
%\input{version-history/6-4.history.tex}
%\input{version-history/6-3.history.tex}
%\input{version-history/6-2.history.tex}
%\input{version-history/6-1.history.tex}
%\input{version-history/6-0.history.tex}
