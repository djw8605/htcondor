\section{\label{sec:clouds-introduction}Introduction}

To be clear, our concern throughout this chapter is with commercial services
which rent computational resources over the Internet at short notice and
charge in small increments (by the minute or the hour).  In 2016, the four
largest such services\footnote{That is, ``infrastructure-as-a-service''
providers.} were (in alphabetical order) Amazon Web Services (`AWS'),
(Microsoft) Azure, Google Cloud Platform (`GCP'), and (IBM) SoftLayer; in
verson 8.7.1, the \Condor{annex} tool supports only AWS.  AWS can start
booting a new virtual machine as quickly as a few seconds after the request;
barring hardware failure, you will be able to continue renting that VM until
you stop paying the hourly charge.  The other cloud services are broadly
similar.

If you already have access to the Grid, you may wonder why you would want to
begin cloud computing.  The cloud services offer two major advantages over
the Grid: first, cloud resources are typically available more quickly and
in greater quantity than from the Grid; and second, because cloud resources are
virtual machines, they are considerably more customizable than Grid resources.
The major disadvantages are, of course, cost and complexity (although we
hope that \Condor{annex} reduces the latter).

We illustrate these advantages with what we anticipate will be the most
common uses for \Condor{annex}.

\subsection{Use Case: Deadlines}

With the ability to acquire computational resources in seconds or minutes
and retain them for days or weeks, it becomes possible to rapidly adjust
the size -- and cost -- of an HTCondor pool.  Giving this ability to the
end-user avoids the problems of deciding who will pay for expanding the
pool and when to do so.  We anticipate that the usual cause for doing so
will be deadlines; the end-user has the best knowledge of their own
deadlines and how much, in monetary terms, it's worth to complete their
work by that deadline.

% We anticipate that the user will want to use images prepared (or
% selected) by their HTCondor pool's administrator for compatiblity with
% the existing pool.

\subsection{Use Case: Capabilities}

Cloud services may offer (virtual) hardware in configurations unavailable in
the local pool, or in quantities that it would be prohibitively expensive to
provide on an on-going basis.  Examples in 2017 may include GPU-based
computation, or computations requiring a terabyte of main memory.  A cloud
service may also offer fast and cloud-local storage for shared data, which
may have substantial performance benefits for some workflows.  Some cloud
providers (for example, AWS) have pre-populated this storage with common
public datasets, to further ease adoption.

By using cloud resources, an HTCondor pool administrator may also experiment
with or temporarily offer different software and configurations.  For
example, a pool may be configured with a maximum job runtime, perhaps to
reduce the latency of fair-share adjustments or to protect against hung
jobs.  Adding cloud resources which permit longer-running jobs may be the
least-disruptive way to accomodate a user whose jobs need more time.

\subsection{Use Case: Capacities}

It may be possible for an HTCondor administrator to lower the cost of their
pool by increasing utilization and meeting peak demand with cloud computing.
