\begin{ManPage}{\label{man-condor-fetchlog}\Condor{fetchlog}}{1}
{Retrieve a daemon's log file that is located on another computer}

\Synopsis 
\SynProg{\Condor{fetchlog}}
\ToolArgsBase

\SynProg{\Condor{fetchlog}}
\oOptArg{-pool}{centralmanagerhostname[:portnumber]}
\oOptnm{-master $|$ -startd $|$ -schedd $|$ -collector $|$ -negotiator
        $|$ -kbdd}
\Arg{machine-name}
\Arg{subsystem[.extension]}

\index{HTCondor commands!condor\_fetchlog}
\index{condor\_fetchlog command}

\Description 

\Condor{fetchlog} contacts HTCondor running on the machine specified by
\Arg{machine-name}, and asks it to return a log file from that
machine.  Which log file is determined from
the \Arg{subsystem[.extension]} argument.
The log file is printed to standard output.
This command eliminates the need to remotely log in to a
machine in order to retrieve a daemon's log file.

For security purposes of authentication and authorization, 
this command requires \Expr{ADMINISTRATOR} level of access.

The \Arg{subsystem[.extension]} argument is utilized to construct
the log file's name.
Without an optional \Arg{.extension},
the value of the configuration variable named \Arg{subsystem}\_LOG 
defines the log file's name.
When specified, the \Arg{.extension} is appended to this value.

The \Arg{subsystem} argument is any value \MacroUNI{SUBSYSTEM}
that has a defined configuration variable of
\Expr{\$(SUBSYSTEM)\_LOG}, or any of
\begin{itemize}
\item \Expr{NEGOTIATOR\_MATCH}
\item \Expr{HISTORY}
\item \Expr{STARTD\_HISTORY}
\end{itemize}

A value for the optional \Arg{.extension} to the \Arg{subsystem} argument
is typically one of the three strings:
\begin{enumerate}
\item{\File{.old}}
\item{\File{.slot<X>}}
\item{\File{.slot<X>.old}}
\end{enumerate}
Within these strings, \verb@<X>@ is substituted with the slot number.
 
A \Arg{subsystem} argument of \Expr{STARTD\_HISTORY} fetches all 
\Condor{startd} history by concatenating all instances of log files
resulting from rotation.

\begin{Options}
    \ToolArgsBaseDesc
    \OptItem{\OptArg{-pool}{centralmanagerhostname[:portnumber]}}
    {Specify a pool by giving the central manager's host name
    and an optional port number}
    \OptItem{\Opt{-master}}{Send the command to the \Condor{master} daemon (default)}
    \OptItem{\Opt{-startd}}{Send the command to the \Condor{startd} daemon}
    \OptItem{\Opt{-schedd}}{Send the command to the \Condor{schedd} daemon}
    \OptItem{\Opt{-collector}}{Send the command to the \Condor{collector} daemon}
    \OptItem{\Opt{-kbdd}}{Send the command to the \Condor{kbdd} daemon}
\end{Options}

\Examples
To get the \Condor{negotiator} daemon's log from a host named 
\File{head.example.com} from within the current pool:
\begin{verbatim}
condor_fetchlog head.example.com NEGOTIATOR
\end{verbatim}

To get the \Condor{startd} daemon's log from a host named
\File{execute.example.com} from within the current pool:
\begin{verbatim}
condor_fetchlog execute.example.com STARTD
\end{verbatim}

This command requested the \Condor{startd} daemon's log from the
\Condor{master}.
If the \Condor{master} has crashed or is unresponsive,
ask another daemon
running on that computer to return the log.
For example, ask the \Condor{startd} daemon to return the
\Condor{master}'s log:

\begin{verbatim}
condor_fetchlog -startd execute.example.com MASTER
\end{verbatim}

\ExitStatus
\Condor{fetchlog} will exit with a status value of 0 (zero) upon success,
and it will exit with the value 1 (one) upon failure.

\end{ManPage}
