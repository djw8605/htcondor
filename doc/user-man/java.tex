
%%%%%%%%%%%%%%%%%%%%%%%%%%%%%%%%%%%%%%%%%%%%%%%%%%%%%%%%%%%%%%%%%%%%%%
\section{\label{sec:Java}Java Applications}
%%%%%%%%%%%%%%%%%%%%%%%%%%%%%%%%%%%%%%%%%%%%%%%%%%%%%%%%%%%%%%%%%%%%%%
\index{Java}

HTCondor allows users to access a wide variety of
machines distributed around the world.
The Java Virtual Machine (JVM)
\index{Java Virtual Machine}
\index{JVM}
provides a uniform platform on any machine, regardless of the
machine's architecture or operating system.
The HTCondor Java universe brings together these
two features to create a distributed, homogeneous computing environment.

Compiled Java programs can be submitted to HTCondor, and HTCondor
can execute the programs on any machine in the pool that will run
the Java Virtual Machine.


The \Condor{status} command can be used to see a list of
machines in the pool for which HTCondor can use the Java Virtual
Machine.

\footnotesize
\begin{verbatim}
% condor_status -java

Name               JavaVendor Ver    State     Activity LoadAv  Mem  ActvtyTime

adelie01.cs.wisc.e Sun Micros 1.6.0_ Claimed   Busy     0.090   873  0+00:02:46
adelie02.cs.wisc.e Sun Micros 1.6.0_ Owner     Idle     0.210   873  0+03:19:32
slot10@bio.cs.wisc Sun Micros 1.6.0_ Unclaimed Idle     0.000   118  7+03:13:28
slot2@bio.cs.wisc. Sun Micros 1.6.0_ Unclaimed Idle     0.000   118  7+03:13:28
...
\end{verbatim}
\normalsize

If there is no output from the
\Condor{status} command,
then HTCondor does not know the location details of the Java Virtual
Machine on machines in the pool,
or no machines have Java correctly installed.
In this case,
contact your system administrator or see section \ref{sec:java-install}
for more information on getting HTCondor to work together
with Java.

%%%%%
\subsection{A Simple Example Java Application}
%%%%%
\index{Java!job example}

Here is a complete, if simple, example.
Start with a simple Java program, \File{Hello.java}:

\begin{verbatim}
public class Hello {
        public static void main( String [] args ) {
                System.out.println("Hello, world!\n");
        }
}
\end{verbatim}

Build this program using your Java compiler.
On most platforms, this is
accomplished with the command
\begin{verbatim}
javac Hello.java
\end{verbatim}

Submission to HTCondor requires a submit description file.
If submitting where files are accessible using a
shared file system,
this simple submit description file works:

\begin{verbatim}
  ####################
  #
  # Example 1
  # Execute a single Java class
  #
  ####################

  universe       = java
  executable     = Hello.class
  arguments      = Hello
  output         = Hello.output
  error          = Hello.error
  queue
\end{verbatim}

The Java universe must be explicitly selected.

The main class of the program is given in the \SubmitCmd{executable} statement.
This is a file name which contains the entry point of the program.
The name of the main class (not a file name) must
be specified as the first argument to the program.

If submitting the job where a shared file system is \emph{not}
accessible,
the submit description file becomes:

\begin{verbatim}
  ####################
  #
  # Example 2
  # Execute a single Java class,
  # not on a shared file system
  #
  ####################

  universe       = java
  executable     = Hello.class
  arguments      = Hello
  output         = Hello.output
  error          = Hello.error
  should_transfer_files = YES
  when_to_transfer_output = ON_EXIT
  queue
\end{verbatim}
For more information about using HTCondor's file transfer mechanisms,
see section~\ref{sec:file-transfer}.

To submit the job, where the submit description file
is named \File{Hello.cmd}, 
execute 
\begin{verbatim}
condor_submit Hello.cmd
\end{verbatim}

To monitor the job, the commands \Condor{q} and \Condor{rm}
are used as with all jobs.

%%%%%
\subsection{Less Simple Java Specifications}
%%%%%

\begin{description}
\item[Specifying more than 1 class file.]
\index{Java!multiple class files}
For programs that 
consist of more than one \Code{.class} file,
identify the files in the submit description file:

\begin{verbatim}
executable = Stooges.class
transfer_input_files = Larry.class,Curly.class,Moe.class
\end{verbatim}

The \SubmitCmd{executable} command does not change.
It still identifies the class file that contains the program's
entry point.

\item[JAR files.]
\index{Java!using JAR files}
If the program consists of a large number of class files,
it may be easier to collect them all together into
a single Java Archive (JAR) file.
A JAR can be created with:

\footnotesize
\begin{verbatim}
% jar cvf Library.jar Larry.class Curly.class Moe.class Stooges.class
\end{verbatim}
\normalsize

HTCondor must then be told where to find the JAR as well
as to use the JAR. 
The JAR file that contains the entry point
is specified with the \SubmitCmd{executable} command.
All JAR files are specified with the \SubmitCmd{jar\_files}
command.
For this example that collected all the class files
into a single JAR file, the submit description file contains:

\begin{verbatim}
executable = Library.jar
jar_files = Library.jar
\end{verbatim}

Note that the JVM must know whether it is receiving JAR files
or class files.
Therefore, HTCondor must also be informed, in order to pass the
information on to the JVM.
That is why there is a difference in submit description file commands
for the two ways of specifying files (\SubmitCmd{transfer\_input\_files}
and \SubmitCmd{jar\_files}).

If there are multiple JAR files,
the \SubmitCmd{executable} command specifies the JAR file
that contains the program's entry point.
This file is also listed with the \SubmitCmd{jar\_files} command:
\begin{verbatim}
executable = sortmerge.jar
jar_files = sortmerge.jar,statemap.jar
\end{verbatim}

\item[Using a third-party JAR file.]
As HTCondor requires that all JAR files (third-party or not)
be available,
specification of a third-party JAR file is no different than
other JAR files.
If the sortmerge example above also relies on
version 2.1 from http://jakarta.apache.org/commons/lang/,
and this JAR file has been placed in the same directory with
the other JAR files, then the submit description file contains
\footnotesize
\begin{verbatim}
executable = sortmerge.jar
jar_files = sortmerge.jar,statemap.jar,commons-lang-2.1.jar
\end{verbatim}
\normalsize

\item[An executable JAR file.]
When the JAR file is an executable, 
specify the program's entry point in the \SubmitCmd{arguments}
command:
\begin{verbatim}
executable = anexecutable.jar
jar_files  = anexecutable.jar
arguments  = some.main.ClassFile
\end{verbatim}

\item[Discovering the main class within a JAR file.]
As of Java version 1.4, 
Java virtual machines have a \Opt{-jar} option,
which takes a single JAR file as an argument.
With this option, 
the Java virtual machine discovers the main class to run 
from the contents of the Manifest file,
which is bundled within the JAR file.  
HTCondor's \SubmitCmd{java} universe does not support this discovery,
so before submitting the job,
the name of the main class must be identified.

For a Java application which is run on the command line with

\begin{verbatim}
  java -jar OneJarFile.jar
\end{verbatim}

the equivalent version after discovery might look like

\begin{verbatim}
  java -classpath OneJarFile.jar TheMainClass
\end{verbatim}

The specified value for \verb@TheMainClass@
can be discovered by unjarring the JAR file,
and looking for the MainClass definition in the Manifest file.
Use that definition in the HTCondor submit description file.
Partial contents of that file Java universe submit file will appear as

\begin{verbatim}
  universe   = java
  executable =  OneJarFile.jar
  jar_files = OneJarFile.jar
  Arguments = TheMainClass More-Arguments
  queue 
\end{verbatim}


\item[Packages.]
\index{Java!using packages}
An example of a Java class that is declared in a non-default
package is
\begin{verbatim}
package hpc;

 public class CondorDriver
 {
     // class definition here
 }
\end{verbatim}
The JVM needs to know the location of this package.
It is passed as a command-line argument, implying the use
of the naming convention and directory structure.

Therefore, the submit description file for this example will contain
\begin{verbatim}
arguments = hpc.CondorDriver
\end{verbatim}

\item[JVM-version specific features.]
If the program uses Java features found only in certain
JVMs, then the Java application submitted to HTCondor
must only run on those machines within the
pool that run the needed JVM.
Inform HTCondor by adding a \Code{requirements}
statement to the submit description file.
For example, to require version 3.2, add to the submit description
file:

\begin{verbatim}
requirements = (JavaVersion=="3.2")
\end{verbatim}

\item[Benchmark speeds.]
Each machine with Java capability in an HTCondor pool
will execute a benchmark to determine its speed.
The benchmark is taken when HTCondor is started on
the machine, and it uses the SciMark2
(\URL{http://math.nist.gov/scimark2}) benchmark.
The result of the benchmark is held as an attribute
within the 
machine ClassAd.
The attribute is called \AdAttr{JavaMFlops}.
Jobs that are run under the Java universe (as all other HTCondor jobs)
may prefer or require a machine of a specific speed
by setting \AdAttr{rank} or \AdAttr{requirements} in
the submit description file.
As an example, to execute only on machines of a minimum speed:

\begin{verbatim}
requirements = (JavaMFlops>4.5)
\end{verbatim}

\item[JVM options.]
Options to the JVM itself are specified in the 
submit description file:

\begin{verbatim}
java_vm_args = -DMyProperty=Value -verbose:gc -Xmx1024m
\end{verbatim}

These options are those which go after the java command, but before
the user's main class.  Do not use this to set the classpath, as
HTCondor handles that itself.  Setting these options is useful for
setting system properties, system assertions and debugging certain
kinds of problems.

\end{description}

%%%%%
\subsection{Chirp I/O}
%%%%%

\index{Chirp}
If a job has more sophisticated I/O requirements that cannot
be met by HTCondor's file transfer mechanism,
then the Chirp facility may provide a solution.
Chirp has two advantages over simple, whole-file transfers.
First, it permits the input files to be decided upon at run-time
rather than submit time, and second,
it permits partial-file I/O with results than can be seen as the
program executes.
However, small changes to the program are required
in order to take advantage of Chirp.
Depending on the style of the program, use either Chirp I/O streams
or UNIX-like I/O functions.

\index{Chirp!ChirpInputStream}
\index{Chirp!ChirpOutputStream}
Chirp I/O streams are the easiest way to get started.
Modify the program to use the objects \Code{ChirpInputStream}
and \Code{ChirpOutputStream} instead of \Code{FileInputStream} and
\Code{FileOutputStream}.
These classes are completely documented
\index{Software Developer's Kit!Chirp}
\index{SDK!Chirp}
in the HTCondor Software Developer's Kit (SDK).
Here is a simple code example:

\begin{verbatim}
import java.io.*;
import edu.wisc.cs.condor.chirp.*;

public class TestChirp {

   public static void main( String args[] ) {

      try {
         BufferedReader in = new BufferedReader(
            new InputStreamReader(
               new ChirpInputStream("input")));

         PrintWriter out = new PrintWriter(
            new OutputStreamWriter(
               new ChirpOutputStream("output")));

         while(true) {
            String line = in.readLine();
            if(line==null) break;
            out.println(line);
         }
         out.close();
      } catch( IOException e ) {
         System.out.println(e);
      }
   }
}
\end{verbatim}

\index{Chirp!ChirpClient}
To perform UNIX-like I/O with Chirp,
create a \Code{ChirpClient} object.
This object supports familiar operations such as \Code{open}, \Code{read},
\Code{write}, and \Code{close}.
Exhaustive detail of the methods may be found in the HTCondor 
SDK, but here is a brief example:

\begin{verbatim}
import java.io.*;
import edu.wisc.cs.condor.chirp.*;

public class TestChirp {

   public static void main( String args[] ) {

      try {
         ChirpClient client = new ChirpClient();
         String message = "Hello, world!\n";
         byte [] buffer = message.getBytes();

         // Note that we should check that actual==length.
         // However, skip it for clarity.

         int fd = client.open("output","wct",0777);
         int actual = client.write(fd,buffer,0,buffer.length);
         client.close(fd);

         client.rename("output","output.new");
         client.unlink("output.new");

      } catch( IOException e ) {
         System.out.println(e);
      }
   }
}
\end{verbatim}

\index{Chirp!Chirp.jar}
Regardless of which I/O style, 
the Chirp library must be specified and included with the job.
The Chirp JAR (\Code{Chirp.jar})
is found in the \File{lib} directory of the HTCondor installation.
Copy it into your working directory in order to
compile the program after modification to use Chirp I/O.

\begin{verbatim}
% condor_config_val LIB
/usr/local/condor/lib
% cp /usr/local/condor/lib/Chirp.jar .
\end{verbatim}

Rebuild the program with the Chirp JAR file in the class path.

\begin{verbatim}
% javac -classpath Chirp.jar:. TestChirp.java
\end{verbatim}

The Chirp JAR file must be specified in the submit description file.
Here is an example submit description file that works for both
of the given test programs:

\begin{verbatim}
universe = java
executable = TestChirp.class
arguments = TestChirp
jar_files = Chirp.jar
+WantIOProxy = True
queue
\end{verbatim}
