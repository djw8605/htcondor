%%%%%%%%%%%%%%%%%%%%%%%%%%%%%%%%%%%%%%%%%%%%%%%%%%%%%%%%%%%%%%%%%%%%%%
\section{Time Scheduling for Job Execution}
\label{sec:Job-Executetime-Scheduling}
%%%%%%%%%%%%%%%%%%%%%%%%%%%%%%%%%%%%%%%%%%%%%%%%%%%%%%%%%%%%%%%%%%%%%%
\index{scheduling jobs!to execute at a specific time}
\index{job execution!at a specific time}

Jobs may be scheduled to begin execution at a specified time in the future
with HTCondor's job deferral functionality.
All specifications are in a job's submit description file.
Job deferral functionality is expanded to provide for the
periodic execution of a job, known as the CronTab scheduling.

%%%%%%%%%%%%%%%%%%%%%%%%%%%%%%%%%%%%%%%%%%%
\subsection{Job Deferral}
\label{sec:JobDeferral}
%%%%%%%%%%%%%%%%%%%%%%%%%%%%%%%%%%%%%%%%%%%
\index{job deferral time}
\index{deferral time!of a job}

Job deferral allows the specification of
the exact date and time at which a job is to begin executing.
HTCondor attempts to match the job to an execution machine
just like any other job,
however, the job will wait until the exact time to begin execution.
A user can define the job to allow some flexibility in the execution of jobs
that miss their execution time.

%%%%%%%%%%%%%%%%%%%%%%%%%%%%%%%%%%%%%%%%%%%
\subsubsection{Deferred Execution Time}
\label{sec:JobDeferral-DeferralTime}
%%%%%%%%%%%%%%%%%%%%%%%%%%%%%%%%%%%%%%%%%%%
\index{deferral time!of a job}
\index{ClassAd job attribute!DeferralTime}

A job's deferral time is the exact time that HTCondor should attempt
to execute the job.
The deferral time attribute is defined as an expression
that evaluates to a Unix Epoch timestamp
(the number of seconds elapsed since 00:00:00 on January 1, 1970,
Coordinated Universal Time).
This is the time that HTCondor will begin to execute the job.

After a job is matched and all of its files have been transferred
to an execution machine,
HTCondor checks to see if the job's ClassAd contains a deferral time.
If it does,
HTCondor calculates the number of seconds between the execution
machine's current system time and the job's deferral time.
If the deferral time is in the future,
the job waits to begin execution.
While a job waits,
its job ClassAd attribute \AdAttr{JobStatus} indicates the job
is in the Running state.
As the deferral time arrives, the job begins to execute.
If a job misses its execution time,
that is, if the deferral time is in the past,
the job is evicted from the execution machine and put on hold in the queue.

The specification of a deferral time does not interfere
with HTCondor's behavior.
For example, if a job is waiting to begin execution
when a \Condor{hold} command is issued,
the job is removed from the execution machine and is put on hold.
If a job is waiting to begin execution when 
a \Condor{suspend} command is issued,
the job continues to wait.
When the deferral time arrives,
HTCondor begins execution for the job,
but immediately suspends it.

The deferral time is specified in the job's submit description file
with the command \SubmitCmd{deferral\_time}.

%%%%%%%%%%%%%%%%%%%%%%%%%%%%%%%%%%%%%%%%%%%
\subsubsection{Deferral Window}
\label{sec:JobDeferral-DeferralWindow}
%%%%%%%%%%%%%%%%%%%%%%%%%%%%%%%%%%%%%%%%%%%
\index{ClassAd job attribute!DeferralWindow}
\index{submit commands!deferral\_window}

If a job arrives at its execution machine
after the deferral time has passed,
the job is evicted from the machine and put on hold in the job queue.
This may occur, for example,
because the transfer of needed files took too long
due to a slow network connection.
A deferral window permits the execution of a job
that misses its deferral time by specifying a window of
time within which the job may begin.

The deferral window 
is the number of seconds after the deferral time,
within which the job may begin.
When a job arrives too late,
HTCondor calculates the difference in seconds
between the execution machine's current time
and the job's deferral time.
If this difference is less than or equal to the deferral window,
the job immediately begins execution.
If this difference is greater than the deferral window,
the job is evicted from the execution machine
and is put on hold in the job queue.

The deferral window is specified in the job's submit description file
with the command \SubmitCmd{deferral\_window}.

%%%%%%%%%%%%%%%%%%%%%%%%%%%%%%%%%%%%%%%%%%%
\subsubsection{Preparation Time}
\label{sec:JobDeferral-PrepTime}
%%%%%%%%%%%%%%%%%%%%%%%%%%%%%%%%%%%%%%%%%%%
\index{ClassAd job attribute!DeferralPrepTime}

When a job defines a deferral time far in the future and then 
is matched to an execution machine,
potential computation cycles are lost because the deferred job
has claimed the machine, but is not actually executing. 
Other jobs could execute during the interval when the job 
waits for its deferral time.
To make use of the wasted time,
\index{submit commands!deferral\_prep\_time}
a job defines a \SubmitCmd{deferral\_prep\_time}
with an integer expression that evaluates to a
number of seconds.
At this number of seconds before the deferral time,
the job may be matched with a machine.

%%%%%%%%%%%%%%%%%%%%%%%%%%%%%%%%%%%%%%%%%%%
\subsubsection{Usage Examples}
\label{sec:JobDeferral-Examples}
%%%%%%%%%%%%%%%%%%%%%%%%%%%%%%%%%%%%%%%%%%%

\index{submit commands!deferral\_time}
Here are examples of how the job deferral time,
deferral window, and the preparation time may be used.

The job's submit description file specifies that
the job is to begin execution 
on January 1st, 2006 at 12:00 pm:

\begin{verbatim} 
   deferral_time = 1136138400
\end{verbatim} 

The Unix \Prog{date} program may be used to calculate
a Unix epoch time.
The syntax of the command to do this depends on the options provided
within that flavor of Unix.  In some, it appears as
\begin{verbatim} 
%  date --date "MM/DD/YYYY HH:MM:SS" +%s
\end{verbatim} 
and in others, it appears as 
\begin{verbatim} 
%  date -d "YYYY-MM-DD HH:MM:SS" +%s
\end{verbatim} 

\verb@MM@ is a 2-digit month number,
\verb@DD@ is a 2-digit day of the month number, and
\verb@YYYY@ is a 4-digit year.
\verb@HH@ is the 2-digit hour of the day,
\verb@MM@ is the 2-digit minute of the hour, and
\verb@SS@ are the 2-digit seconds within the minute.
The characters \verb@+%s@ tell the \Prog{date} program
to give the output as a Unix epoch time.

The job always waits 60 seconds before
beginning execution:

\begin{verbatim} 
   deferral_time = (time() + 60)
\end{verbatim}

In this example, assume that the deferral time is 45 seconds
in the past as the job is available.
The job begins execution, because 75 seconds remain in the
deferral window:

\begin{verbatim} 
   deferral_window = 120
\end{verbatim}

In this example, a job is scheduled to execute
far in the future,
on January 1st, 2010 at 12:00 pm. 
The \SubmitCmd{deferral\_prep\_time} attribute delays the job 
from being matched until 60 seconds before the job is to begin execution. 

\begin{verbatim}
   deferral_time      = 1262368800
   deferral_prep_time = 60
\end{verbatim}

%%%%%%%%%%%%%%%%%%%%%%%%%%%%%%%%%%%%%%%%%%%
\subsubsection{Limitations}
\label{sec:JobDeferral-Limitations}
%%%%%%%%%%%%%%%%%%%%%%%%%%%%%%%%%%%%%%%%%%%
There are some limitations to HTCondor's job deferral feature.

\begin{itemize}
\item Job deferral is not available for scheduler universe jobs.
% no referring to daemons in the user's manual!
% Scheduler universe jobs are not executed under the control 
% of the \Condor{starter} daemon, 
% which is needed to defer the job until the correct execution time. 
A scheduler universe job defining the \AdAttr{deferral\_time}
produces a fatal error when submitted.

\item The time that the job begins to execute 
is based on the execution machine's system clock, 
and not the submission machine's system clock. 
Be mindful of the ramifications when
the two clocks show dramatically different times.

\item A job's \AdAttr{JobStatus} attribute is always in the Running state 
when job deferral is used.
There is currently no way to distinguish between a job that is 
executing and a job that is waiting for its deferral time. 

\end{itemize}

%%%%%%%%%%%%%%%%%%%%%%%%%%%%%%%%%%%%%%%%%%%
\subsection{CronTab Scheduling}
\label{sec:CronTab}
%%%%%%%%%%%%%%%%%%%%%%%%%%%%%%%%%%%%%%%%%%%
\index{CronTab job scheduling}
\index{job scheduling!periodic}
\index{scheduling jobs!to execute periodically}

HTCondor's CronTab scheduling functionality allows jobs to be 
scheduled to execute periodically. 
A job's execution schedule is defined by commands within
the submit description file.
The notation is much like that used by the Unix \Prog{cron} daemon. 
As such, HTCondor developers are fond of referring to CronTab
\index{Crondor}
scheduling as \Term{Crondor}.
The scheduling of jobs using HTCondor's CronTab feature 
calculates and utilizes
the \Attr{DeferralTime} ClassAd attribute. 

Also, unlike the Unix \Prog{cron} daemon, 
HTCondor never runs more than one instance of a job at the same time. 

The capability for repetitive or periodic execution of the job is 
enabled by specifying an \SubmitCmd{on\_exit\_remove}
command for the job,
such that the job does not leave the queue until desired.

%%%%%%%%%%%%%%%%%%%%%%%%%%%%%%%%%%%%%%%%%%%
\subsubsection{Semantics for CronTab Specification}
\label{sec:CronTab-Semantics}
%%%%%%%%%%%%%%%%%%%%%%%%%%%%%%%%%%%%%%%%%%%

A job's execution schedule is defined by a set of specifications
within the submit description file.
HTCondor uses these to calculate a \Attr{DeferralTime} for the job.

Table \ref{tab:CronTab-Attributes} 
lists the submit commands and acceptable values for these commands.
At least one of these must be defined 
in order for HTCondor to calculate a \Attr{DeferralTime} for the job.
Once one CronTab value is defined, 
the default for all the others uses 
all the values in the allowed values ranges.

\index{submit commands!cron\_minute}
\index{submit commands!cron\_hour}
\index{submit commands!cron\_day\_of\_month}
\index{submit commands!cron\_month}
\index{submit commands!cron\_day\_of\_week}

\begin{table}
   \begin{center}
   \begin{tabular}{ll}
   Submit Command & Allowed Values \\
   \hline
   \SubmitCmdNI{cron\_minute} & 0 - 59 \\
   \SubmitCmdNI{cron\_hour} & 0 - 23 \\
   \SubmitCmdNI{cron\_day\_of\_month} & 1 - 31 \\
   \SubmitCmdNI{cron\_month} & 1 - 12 \\
   \SubmitCmdNI{cron\_day\_of\_week} & 0 - 7 (Sunday is 0 or 7)\\
   \end{tabular}
   \end{center}
   \caption{The list of submit commands and their value ranges.}
   \label{tab:CronTab-Attributes}
\end{table}

The day of a job's execution can be specified 
by both the \SubmitCmdNI{cron\_day\_of\_month} 
and the \SubmitCmdNI{cron\_day\_of\_week} attributes. 
The day will be the logical or of both.

The semantics allow more than one value to be specified 
by using the \verb@*@ operator,
ranges, lists, and steps (strides) within ranges.

\begin{description}
   \item[The asterisk operator]
   The \verb@*@ (asterisk) operator specifies that all of the 
   allowed values are used for scheduling.
   For example,
   \begin{verbatim}
      cron_month = *
   \end{verbatim}
   becomes any and all of the list of possible months:
   (1,2,3,4,5,6,7,8,9,10,11,12).
   Thus, a job runs any month in the year.

   \item[Ranges]
   A range creates a set of integers from all the allowed values between two
   integers separated by a hyphen. The specified range is inclusive, and the
   integer to the left of the hyphen must be less than the right hand integer.
   For example,
   \begin{verbatim}
      cron_hour = 0-4
   \end{verbatim}
   represents the set of
   hours from 12:00 am (midnight) to 4:00 am, or (0,1,2,3,4).
   
   \item[Lists]
   A list is the union of the values or ranges separated by commas. Multiple
   entries of the same value are ignored. 
   For example,
   \begin{verbatim}
      cron_minute = 15,20,25,30
      cron_hour   = 0-3,9-12,15
   \end{verbatim}
   where this \SubmitCmdNI{cron\_minute} example represents (15,20,25,30)
   and \SubmitCmdNI{cron\_hour} represents (0,1,2,3,9,10,11,12,15).
      
   \item[Steps]
   Steps select specific numbers from a range, based on an interval.
   A step is specified by appending a range or the asterisk
   operator with a slash character (\verb@/@),
   followed by an integer value.
   For example,
   \begin{verbatim}
      cron_minute = 10-30/5
      cron_hour = */3
   \end{verbatim}
   where this \SubmitCmdNI{cron\_minute} example specifies
   every five minutes within the specified range 
   to represent (10,15,20,25,30),
   and \SubmitCmdNI{cron\_hour} specifies every three hours of the day
   to represent (0,3,6,9,12,15,18,21).
   

\end{description}

%%%%%%%%%%%%%%%%%%%%%%%%%%%%%%%%%%%%%%%%%%%
\subsubsection{Preparation Time and Execution Window}
\label{sec:CronTab-PrepTime}
%%%%%%%%%%%%%%%%%%%%%%%%%%%%%%%%%%%%%%%%%%%

The \SubmitCmd{cron\_prep\_time} command
is analogous to the deferral time's \SubmitCmd{deferral\_prep\_time} command. 
It specifies the number of seconds before the deferral time
that the job is to be matched and sent to the execution machine. 
This permits HTCondor to
make necessary preparations before the deferral time occurs. 

Consider the submit description file example that includes 
\begin{verbatim}
   cron_minute = 0
   cron_hour = *
   cron_prep_time = 300
\end{verbatim}
The job is scheduled to begin execution at the top of every hour.
Note that the setting of \SubmitCmdNI{cron\_hour} in this example
is not required, as the default value will be \verb@*@, 
specifying any and every hour of the day.
The job will be matched and sent to an execution machine 
no more than five minutes before the next deferral time. 
For example, if a job is submitted at 9:30am, then the 
next deferral time will be calculated to be 10:00am.
HTCondor may attempt to match the job to a machine and send the job
once it is 9:55am.

As the CronTab scheduling calculates and uses deferral time,
jobs may also make use of the deferral window.
The submit command \SubmitCmd{cron\_window} is analogous to
the submit command \SubmitCmd{deferral\_window}.
Consider the submit description file example that includes 
\begin{verbatim}
   cron_minute = 0
   cron_hour = *
   cron_window = 360
\end{verbatim}
As the previous example, the job is scheduled to begin execution
at the top of every hour.
Yet with no preparation time, the job is likely to miss
its deferral time.
The 6-minute window allows the job to begin execution,
as long as it arrives and can begin within 6 minutes of
the deferral time,
as seen by the time kept on the execution machine.

%%%%%%%%%%%%%%%%%%%%%%%%%%%%%%%%%%%%%%%%%%%
\subsubsection{Scheduling}
\label{sec:crontab-scheduling}
%%%%%%%%%%%%%%%%%%%%%%%%%%%%%%%%%%%%%%%%%%%

When a job using the CronTab functionality is submitted to HTCondor, 
use of at least one of the submit description file commands
beginning with \SubmitCmdNI{cron\_} causes HTCondor
to calculate and set a deferral time for when the job should run. 
A deferral time is determined based on the current time 
rounded later in time to the next minute. 
The deferral time is the job's \AdAttr{DeferralTime} attribute. 
A new deferral time is calculated when the job 
first enters the job queue, when 
the job is re-queued, or when the job is released from the hold state. 
New deferral times for \emph{all} jobs in the job queue 
using the CronTab functionality are recalculated 
when a \Condor{reconfig} or a \Condor{restart} command that
affects the job queue is issued.

A job's deferral time is not always the same time that a job 
will receive a match and be sent to the execution machine. 
This is because HTCondor operates on the job queue
at times that are independent of job events,
such as when job execution completes.
Therefore,
HTCondor may operate on the job queue just after 
a job's deferral time states that it is to begin execution. 
HTCondor attempts to start a job when the 
following pseudo-code boolean expression evaluates to \Expr{True}:

\footnotesize
\begin{verbatim}
   ( time() + SCHEDD_INTERVAL ) >= ( DeferralTime - CronPrepTime )
\end{verbatim}
\normalsize

If the \Attr{time()} plus the number of seconds 
until the next time HTCondor checks 
the job queue is greater than or equal to the time that the job 
should be submitted to the execution machine, 
then the job is to be matched and sent now.

Jobs using the CronTab functionality are not automatically 
re-queued by HTCondor after their execution is complete. 
The submit description file for a job
must specify an appropriate \SubmitCmd{on\_exit\_remove} 
command to ensure that a job remains in the queue. 
This job maintains its original \Attr{ClusterId} and \Attr{ProcId}.

%%%%%%%%%%%%%%%%%%%%%%%%%%%%%%%%%%%%%%%%%%%
\subsubsection{Usage Examples}
\label{sec:crontab-examples}
%%%%%%%%%%%%%%%%%%%%%%%%%%%%%%%%%%%%%%%%%%%

Here are some examples of the submit commands
necessary to schedule jobs to run at multifarious times. 
Please note that it is not necessary to 
explicitly define each attribute; the default value is \verb@*@.

Run 23 minutes after every two hours, every day of the week:

\begin{verbatim}
   on_exit_remove = false
   cron_minute = 23
   cron_hour = 0-23/2
   cron_day_of_month = *
   cron_month = *
   cron_day_of_week = *
\end{verbatim}

Run at 10:30pm on each of May 10th to May 20th, as well as every 
remaining Monday within the month of May:

\begin{verbatim}
   on_exit_remove = false
   cron_minute = 30
   cron_hour = 20
   cron_day_of_month = 10-20
   cron_month = 5
   cron_day_of_week = 2
\end{verbatim}

Run every 10 minutes and every 6 minutes before noon 
on January 18th with a 2-minute preparation time:

\begin{verbatim}
   on_exit_remove = false
   cron_minute = */10,*/6
   cron_hour = 0-11
   cron_day_of_month = 18
   cron_month = 1
   cron_day_of_week = *
   cron_prep_time = 120
\end{verbatim}

%%%%%%%%%%%%%%%%%%%%%%%%%%%%%%%%%%%%%%%%%%%
\subsubsection{Limitations}
\label{sec:Crontab-Limitations}
%%%%%%%%%%%%%%%%%%%%%%%%%%%%%%%%%%%%%%%%%%%
The use of the CronTab functionality has all of the same 
limitations of deferral times,
because the mechanism is based upon deferral times.

\begin{itemize}
\item It is impossible to schedule vanilla 
and standard universe jobs 
at intervals that are smaller than the
interval at which HTCondor evaluates jobs.
This interval is determined by 
the configuration variable \Macro{SCHEDD\_INTERVAL}. 
As a vanilla or standard universe job completes execution 
and is placed back into the job queue, 
it may not be placed in the idle state in time.
This problem does not afflict local universe jobs.

\item HTCondor cannot guarantee that a job will be
matched in order to make its scheduled deferral time.
A job must be matched with an execution machine just as
any other HTCondor job; 
if HTCondor is unable to find a match, 
then the job will miss its chance for executing
and must wait for the next execution time 
specified by the CronTab schedule.

\end{itemize}
